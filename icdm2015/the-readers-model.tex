In the previous section we proposed a measure of diversity, which can
be applied to a set of distributions
$\cP_{W}=\{P_{w_1},...,P_{w_k}\}$. Additionally, we required that 
each of the distributions be assigned a weight $d_{w_i}$, that
represents the importance of the corresponding word $w_i$ in the given
text. Next, we will discuss the issues involved in building the right
distributional word representations, and propose a new model for
addressing them. As a starting point, we use 
LDA~\cite{Blei:2003:LDA:944919.944937} to build a topic model given a
document corpus $\cC$. We will use $T$ to denote
the set of learned topics. As a result of this procedure, each word
occurence in the corpus is tagged with one of the topics $t\in
T$. From this model we obtain the  
word-topic-count matrix $\cM$ where $\cM_{ij}$ is the number of
assignments of $j$-th topic to word $w_i$ in the corpus. By
normalizing the rows of matrix $\cM$ we can obtain 
a set of distributional representations, where the $i$-th row of the
matrix gives a topic distribution $P_{w_i}$ for the word $w_i$. In
other words, for a topic $t\in T$ we assign $P_{w_i}(t)$ to be the
fraction of occurrences of $w_i$ in the corpus, which were tagged with
topic $t$. That simple approach provides the right intuition for the
type of representations that we are looking for, however it has
certain problems that need to be addressed.

{\bf Sample size bias problem.}  Note that 
if we normalize the rows of the matrix $\cM$, the normalization factor   
for a word $w$ is simply the number of occurrences of $w$ in the
corpus. For a word which occurs very rarely, the topic
distribution will be inaccurate simply because we do not have 
enough data points to estimate its true topic distribution. 
% Now, if for this corpus it happens that the prior topic distribution
% is close to uniform, it will give a very high importance to the word 
% $w$ when measuring the information diversity as described in
% Section~\ref{sec:information-diversity}.
One way to alleviate this problem is to use the relative sample size
(e.g., word count) to smooth out the distribution obtained by
normalizing the rows of matrix $\cM$. 
% A natural choice for the
% smoothing distribution would in this case be simply the prior
% distribution mentioned above.
Applying {\em Laplace smoothing}, we get:
\begin{equation}
\widehat{P}_w=\frac{\alpha P_0+ \mu_w P_w}{\alpha+\mu_w}
\end{equation}
where $\mu_w$ is the total number of occurrences of word $w$ in D, and $P_w$ is the
topic assignment distribution obtained from the word-topic matrix,
while $\alpha$ is the parameter that specifies the strength of the
smoothing prior distribution $P_0$. We obtain a prior 
 topic distribution $P_0$ by computing the proportions of overall topic
 assignments. This corresponds to summing up matrix $\cM$ along its
 columns and then normalizing the resulting vector.

 {\bf Capturing topic correlations.} Here we consider the problem
first raised in~\cite{bache:2013}. When building a topic
distribution model based on the word-to-topic co-occurrence matrix,
the relationship among topics 
(i.e, topic correlation) may be lost. 
For example, a product description that is related to two
topics such as {\em Computers} and {\em Software} should be deemed
less surprising or interesting then if those two topics were {\em
  Computers} and {\em Clothes}, so our model has to be able to make
such a distinction.
%  We need a way to distinguish 
% For example, if a word $w$ 
% has a topic distribution $P_w$ concentrated on some topic $t$, then it will
% not necessarily peak at all other topics that are very similar to
% topic $t$. 
To address this problem, we define a topic similarity
matrix $\cS_{|T|\times|T|}$ 
where the element $S_{i,j}$ gives the similarity/correlation between
$i^{th}$ and $j^{th}$ topic (e.g. the similarity would be high between {\em
  Computers} and {\em Software}, but low for {\em Computers} and
{\em Clothes}). Using
$\cS$ we transform the word-topic-count matrix $\cM$, obtaining
$\cM_\cS = \cM\cS^T$, which effectively diffuses the topic
distributions to ones where for every topic assignment, all similar topics are
also well represented. A more detailed explanation of this step is provided
in Section~\ref{sec:topic-similarities}.

{\bf Context-dependent representations.} We propose a final
enhancement to the topic 
distributions. Suppose, the set of words $W=\{w_1,...,w_k\}$
represents a text snippet that we want to analyze. The word $w_i$ has
a specific meaning inside of $W$, that can be significantly different
than its meaning out of context. For example, the word {\em mouse} can
be associated with the topic of computers as well as the topic of
animals, which are clearly very different. However, in a given
context, usually there will be no confusion as to which topic is
most relevant. We want to define a new context-dependent
distributional representation of $w_i$ that takes this into account
(as opposed to the global representation $\widehat{P}_{w_i}$ we have so far).
We will represent the context information through a topic
distribution $P_c$, which describes the
topics relevant to the words in $W$ (excluding the word in question, i.e., $w_i$), defined later as a mixture of
their distributional representations. If a topic $t$ has high weight in
both distributions $\widehat{P}_{w_i}$ and $P_c$, then it is relevant to $w_i$ in this
context. More generally, if distributions $P_c$ and $\widehat{P}_{w_i}$ have
significant overlap, then those shared topics are likely to explain the
text well, and the remaining topics in $\widehat{P}_{w_i}$ should effectively be
ignored. If the overlap is minimal, then this indicates unusual
(diverse) content. To model this intuition, we define the following
transformation applied to a word representation in a given context.

% Denote
% $W_{\bar{i}}=W-\{w_i\}$ as the set of all words in $W$ except
% $w_i$. By $P_{W_{\bar{i}}}$, we denote the mixture distribution for
% $W_{\bar{i}}$ (Definition \ref{mixture}) and we use
% $\tilde{P}_{W_{\bar{i}}}$ when it is smoothed using Laplace smoothing
% method. We 
% propose the following definition of context-dependent word-topic
% distribution:
\bed\label{context}
Let $P_0,\widehat{P}_{w_i}, P_c$ be the topic prior, global
topic distribution for $w_i$, and the context distribution,
respectively. Then, the context-dependent distribution is
\begin{equation*}
\widehat{P}^{P_c}_{w_i}(t)\propto \left(\frac{P_c(t)}{P_0(t)} + \beta\right)\widehat{P}_{w_i}(t).
\end{equation*}
\eed
This can be intuitively understood as follows:
we can think of $\frac{P_c(t)}{P_0(t)}$ as a weight
that further reshapes the smoothed topic distribution $\widehat{P}_{w_i}(t)$
to take into account the context. The key operation here is the
element-wise multiplication $P_c(t)\cdot\widehat{P}_{w_i}(t)$, which
essentially captures the overlapping topics. The prior $P_0$ is there
to adjust for the overall popularity of each topic - if context
distribution were equal to the prior, we would expect it to have no effect in
this transformation, as is the case here. Note, that we apply additional
Laplace smoothing with parameter $\beta$, to account for the case where
the word and the context have minimal topic overlap, which is very
important because that is what generates diversity. If this occurs,
the transformation effectively reverts to the global representation. Naturally, we still
have to provide a method for obtaining the context distributions, and
also connect the entire procedure to the diversity measure discussed in
Section~\ref{sec:information-diversity}. We will now jointly address
those two questions.


% In the previous section we discussed the general setting leading to
% obtaining distributional representations of data. However, focusing on
% the text domain allows us to extend that model, obtaining more
% accurate distributional representations of words. Recall that we start
% with a text corpus, which can be converted into a word-context
% co-occurrence matrix $\cM_{|V|\times|T|}$. 
% We imagine an entity
% called Reader, that has observed all of this data, trying to learn the
% language from context assignments.
% %  First, the Reader estimates the
% % distributions of words $P_t(w)$ conditioned on each 
% % label $t$, corresponding to the columns of matrix $\Mb$, and also
% % the overall distribution of labels $P(t)$ across the entire dataset of 
% % co-occurrences. 
% % {\bf Do we want to discuss using a prior here? In experiments we don't
% %   do that.}
% We use a simple generative model $(P,\{P_t\}_{t\in T})$ of the corpus: 
% generate a context $t$ from a context distribution $P$, then sample a
% word from the word distribution $P_t$,
% and repeat to obtain all of the words. We let the Reader estimate
% those distributions by maximum likelihood, obtaining the estimated
% model $(\tilde{P},\{\tilde{P}_t\}_{t\in T})$.

% Next, the question is what should be  
% the Reader's distributional representation of a certain word? This
% essentially corresponds to a row of matrix $\cM$. A reasonable choice
% would be to estimate the MAP context distribution conditioned on a word
% $w$ using the appropriate row of $\cM$ and a Dirichlet prior centered
% around $\tilde{P}$. However, we propose
% a more flexible definition, in which one can obtain a different
% distributional representation for the same word, depending on the
% context. Suppose that we are given such context in the form of a context
% distribution $P_c$. As a thought experiment, consider regenerating our
% data using $P_c$ in place of $P$ as the overall context
% distribution (obtaining a different co-occurrence matrix $\cM'$), and
% estimating the MAP context distribution for $w$ again.
% \bep\label{map}
% Given a set of word distributions $\{\tilde{P}_t\}_{t\in T}$ for each
% context, a context
% distribution $P_c$, and a prior context distribution $\widehat{P}$ for a word $w$,
% the MAP context distribution for $w$ in the above model will be
% \[\Pb(t|w,\widehat{P},P_c) \propto \tilde{P}_t(w)\cdot P_c(t) + \alpha
% \widehat{P}(t),\]
% where $\alpha$ is the parameter that specifies the strength of the prior.
% \eep

% This definition has a nice interpretation through the generative
% model: it corresponds to regenerating our data using $P_c$ in place of
% $P$ as the overall label distribution, and computing an MAP label
% distribution conditioned on $w$, with a Dirichlet prior concentrated around
% $\widehat{P}$.
% The most basic application of this is to simply set
% both $P_c$ and $\widehat{P}$ equal to $\tilde{P}$ (i.e. there is no external context
% information). Then, $P(t|w,\tilde{P},\tilde{P})$ is the
% normalized $i$-th row of matrix $\cM$, with added Bayesian
% smoothing. These will be the general distributional word
% representations, denoted $\tilde{P}_w$. 


Let us return to the the task of measuring diversity of a given set of words
$W=\{w_1,...,w_k\}$. Naturally, we aim to use Jensen-Shannon
Divergence, applying it to some mixture of the distributional word
representations. First, let us look at the global representations
$\widehat{P}_{w_i}$ (without context-dependence). How do we choose the
importance weights for the mixture, necessary for using
Definition~\ref{diversity}? 
We 
are tempted to look at the frequencies of each of the words in our
corpus. On the one hand, one may say that an infrequent word is more
discriminative then a common one, because it probably appears in more
specific settings. On the other hand, as we mentioned earlier, a word
that only appears a few 
times in a dataset likely does not have enough occurrences to establish an
accurate representation - in fact Laplace smoothing will force such a
representation to be very close to the prior topic distribution
$P_0$, making it uninformative. Moreover, there is many relatively frequent words
that can be clearly identified with a specific topic (e.g., word
{\em iPhone} corresponding to a relatively narrow product category of smartphones,
appears frequently in the eBay product titles dataset). With
that in mind, we will once again turn to KL-divergence to
estimate how informative a word is with respect to the prior $P_0$
used as a reference distribution.

\bed\label{mixture}
We define the {\bf importance} of a word representation $\widehat{P}_{w_i}$ with
respect to a reference distribution $P_0$ as 
\[D_{w_i}=D_{KL}(\widehat{P}_{w_i}\,\|\,P_0).\]
Given a set of distributions and their importances $\cP =
\{(\widehat{P}_{w_1},D_{w_1}),...,(\widehat{P}_{w_k},D_{w_k})\}$, the
{\bf Informative Mixture} of $\cP$ is a topic distribution defined as 
\[M_{\cP} = \Sigma d_{w_i} \widehat{P}_{w_i},\]
where
\[d_{w_i} = \frac{D_{w_i}}{\sum_{j=1}^k D_{w_j}}.\]
\eed

% We now turn to the contexts themselves, and why a fixed distributional
% word representation may not be sufficient for this task. A simple example will
% help us illustrate this. Consider a word $w_i$ that has
% two separate meanings. In one case, it is always assigned context $t_1$,
% while in the other - context $t_2$. The general distributional
% representation of $w_i$ will put significant weight on both $t_1$ and
% $t_2$, even though in each specific occurrence, $w_i$ represents only one
% or the other. A context distribution $P_c$ could help disambiguate
% the meaning of $w$.
% Note, that the natural choice for the reference distribution $P$ is
% the previously defined prior distribution $P_0$.
The above definition provides a natural way of defining the context
distribution needed for Definition~\ref{context}. 
In our case, the context of $w_i$ will be composed of
all other words in the set $W$, because in our experiments we
focus on short to medium length pieces of text. Otherwise, a moving
window of fixed length may be a more appropriate choice. Let
$\cP^i$ denote the set of global distributional representations
with importances $(\widehat{P}_{w_j},D_{w_j})$ with the $i^{th}$ word
excluded (using prior $P_0$ as the reference
distribution). We use the Informative Mixture 
$M_{\cP^i}$ as a context distribution for the word 
$w_i$, obtaining a new context-dependent representation
$\widehat{P}^{M_{\cP^i}}_{w_i}$. Finally, we can define our diversity
score.
\bed\label{text-diversity}
 Let text $W$ be represented (as explained above) by
\[\cP_c=\{(\widehat{P}^{M_{\cP^1}}_{w_1},D_{w_1}),\dots,
(\widehat{P}^{M_{\cP^k}}_{w_k},D_{w_k})\}.\] 
We will define the
{\bf Jensen-Shannon Text Diversity} of $W$ as
\[\textrm{JSD}_\cM(W) = D_{JS}(M_{\cP_c}).\]
\eed
The whole process of computing text diversity is
presented in Algorithm \ref{alg}. The pseudocode is divided into the
preprocessing steps that can be done on the topic model data, and the
main procedure that returns a diversity score for a given set of words.

\begin{algorithm}
  \caption{Measuring Diversity in Text}
\label{alg}
  \begin{algorithmic} 
  \Procedure{Preprocessing}{$\cM,\cS,\alpha$}
   \State $\cM_\cS \leftarrow \cM\cS^T$
   \State $P_0 \leftarrow $ sum $\cM_\cS$ along the columns, normalize  
   \For {$w \in V$}
    \State $m_w \leftarrow $ the row of $\cM_\cS$ corresponding to $w$
    \State $\mu_w \leftarrow  $ \Call{Sum}{$m_w$}
    \State $\widehat{P}_w\leftarrow (\alpha P_0 + m_w) /
    (\alpha+\mu_w)$
    \State $D_w \leftarrow D_{KL}(\widehat{P}_w\,\|\,P_0)$
   \EndFor
  \EndProcedure
  \Procedure{TextDiversity}{$W,\beta$}
   \For {$w_i\in W$}
    \State $M_{\cP^i} \leftarrow \Sigma d_{w_j} \widehat{P}_{w_j}$ for
    $j\neq i$
    \State $\widehat{P}^{M_{\cP^i}}_{w_i}\leftarrow
    \widehat{P}_{w_i}(\beta\, +\, M_{\cP^i}/P_0)$, then normalize
   \EndFor
   \State\Return $D_{JS}(\Sigma d_{w_j}
   \widehat{P}^{M_{\cP^j}}_{w_j})$
  \EndProcedure
  % \State $B[1..n] \leftarrow [1..n]$ \quad-- \{array of consecutive
  % indices from $1$ to $n$\} 
  % \If {$n = 1$}
  % \If {$A[1]>M$}
  % \State $B[1]\leftarrow 0$ \quad-- \{S is empty, so erasing the index\} 
  % \EndIf
  % \State \Return $B$
  % \EndIf
  % \State $m \leftarrow$ \Call{Median}{$A[1..n]$}
  % \State Partition $A$ w.r.t. $m$ (as in Quick Sort), keeping track of
  % original indices in $B$
  % \State $k \leftarrow$ position of median in $A$ after partitioning
  % \State $N\leftarrow$ \Call{Sum}{$A[1..k]$}
  % \If {$N\leq M$}
  % \State $C\leftarrow $ \Call{LargestSubset}{$A[k+\!1..n],M-N$}
  % \State $B[k+\!1..n]\leftarrow $ [$B[i]$ for $i\in C$ if $i\neq 0$
  % else $0$]
  % \Else 
  % \State $B[k+\!1..n]\leftarrow 0$\quad -- \{erasing all eliminated indices\}
  % \State $C\leftarrow$ \Call{LargestSubset}{$A[1..k],M$}
  % \State $B[1..k]\leftarrow $ [$B[i]$ for $i\in C$ if $i\neq 0$  else
  % $0$] 
  % \EndIf
  % \State\Return $B$
  % \EndProcedure

  \end{algorithmic}
\end{algorithm}


% \bed
% Given a sequence $S=(w_1,...,w_k)$, we define the context-dependent
% distributional representation of word $w_i$ in $S$ by 
% \[\tilde{P}_{w_i}^S(t) = P(t|w_i,\tilde{P}_{w_i},\tilde{P}_S^{w_i}),\]
% where $\tilde{P}_S^{w_i}$ is the Informative Mixture of
% $\{\tilde{P}_{w_j}\}_{j\neq i}$. 
% \eed

% This gives us a new set $\cP_S^c=\{\tilde{P}_{w_1}^S,...,\tilde{P}_{w_k}^S\}$ of
% distributional representations for each word in $S$. Moreover, we can
% use them to obtain a distributional representation of $S$ itself, by
% taking the Informative Mixture $M_{\cP_S^c}$. As we will show in
% experiments, this can be treated separately as a text embedding in
% supervised learning tasks. In this section, however, we will use it to
% propose a notion of text diversity, based on Definition
% \ref{jsd-definition}. 

% \bed
% Given a sequence $S=(w_1,...,w_k)$ and a data model $\cM$, the
% {\bf Jensen-Shannon Text Diversity} will be defined as
% \[\textrm{JSD}_{\cM}(S) = D_{JS}(M_{\cP_S^c}),\] where $\cP^c_S=\{P_{w_1}^S,...,P_{w_k}^S\}$ is the set
% of context-dependent distributional representations for words in $S$. 
% \eed

% TODO: Maybe some Pseudo-Code?

%  We can measure how much information is gained
% in this observation using KL-divergence, obtaining
% $D_{KL}(P_i\|P)$. Moreover, we can bring in some information about
% relations between words, by computing $P_{ij}(\,\cdot\,) =
% P(\,\cdot\,|w_i,P_i,P_j)$, 
% which describes the topic distribution obtained after observing word
% $w_i$ in the context of word $w_j$ (since they were both in
% $S$). Finally, we define a new series of topic
% distributions for each word, which are based on contexts from all other
% words in the sequence, weighed by their information gain.



% \bed
% Given a co-occurrence matrix $\Mb$, we define the Jensen-Shannon
% Information Diversity of a sequence 
% $S=(w_1,...,w_k)$ as
% \[\mbox{JSID}_{\Mb}(S)=D_{JS}(\Sigma d_i \widetilde{P}_i),\]
% where 
% \[d_{i}=\frac{D_{KL}(\widetilde{P}_i\|P_i^c)}{\sum D_{KL}(\widetilde{P}_i\|P_i^c)}.\]
% \eed

% Notice, that we use different weights than were proposed for the
% population diversity (we use surprise instead of perplexity). To
% best observe the difference, let us see what happens when a given word
% provides no information gain (the divergence from context distribution
% is zero). In the above definition, we would put no weight on that
% word, since it is irrelevant to the meaning of sequence $S$. On the
% other hand, using perplexity means that all weights have to be
% non-zero, which makes more sense in the population
% setting. Unfortunately, the
% use of different weights for the information diversity prevents us
% from obtaining the generalization from Proposition
% \ref{generalization}. However, it is easy to see that JSID is still a
% generalization of Shannon entropy in a more narrow sense, because it
% can be interpreted as mutual information. 

% {\bf (4) Sample size bias problem:}  note that 
% when we normalize the rows of the matrix $\cM$ to obtain a word-to-topic distributional model, the normalization factor 
% for every word is simply the word count. Thus for a word $w$ which occurs very rarely
% in the entire corpus $\cC$, the topic distribution will be artificially skewed and is inaccurate simply because we do not have enough data points to estimate its true word-to-topic distribution.
% Now, if for this corpus it happens that the prior topic distribution
% is close to uniform, it will give a very high importance to the word
% $w$ when measuring the information diversity as described in
% Section~\ref{sec:information-diversity}.
% One way to alleviate this
% problem is to use the relative sample size (e.g., word count) to
% smooth the distribution obtained by normalizing the rows of matrix
% $\cM$. A natural choice for the smoothing distribution would in this
% case be simply the prior distribution mentioned above. Applying {\em Laplace smoothing}
% we get:
% \begin{equation}
% \widehat{P}_w=\frac{\alpha \tilde{P}+ \mu_w \tilde{P}_w}{\alpha+\mu_w}
% \end{equation}
% where $\mu_w$ is the frequency of word $w$ in D, and $P_w$ is the
% topic assignment distribution obtained from the word-topic matrix,
% while $\alpha$ is the parameter that specifies the strength of the
% prior.\\
% {\bf (5) Conditioning on context words:} we propose a final enhancement to the word-topic
% distributions. Suppose, the set of words $W=\{w_1,...,w_k\}$
% represents a text snippet that we want to analyze. The word $w_i$ has
% a specific meaning inside of $W$, that can be significantly different
% than its meaning out of context. Denote
% $W_{\bar{i}}=W-\{w_i\}$ as the set of all words in $W$ except
% $w_i$. By $P_{W_{\bar{i}}}$, we denote the mixture distribution for $W_{\bar{i}}$ (Definition \ref{mixture}) and we use $\tilde{P}_{W_{\bar{i}}}$ when it is smoothed using Laplace smoothing method. We
% propose the following definition of context-dependent word-topic
% distribution:
% \bed
% Let $\tilde{P},\widehat{P}_{w_i}, \tilde{P}_{W_{\bar{i}}}$ be the topic prior, general
% topic distribution for $w_i$, and the context distribution,
% respectively. Then, the context-dependent distribution is
% \begin{equation*}
% P^{W_{\bar{i}}}_{w_i}(t)\propto \frac{\tilde{P}_{W_{\bar{i}}}(t)}{\tilde{P}(t)}\widehat{P}_{w_i}(t)
% \end{equation*}
% \eed
% There is a probabilistic explanation that we have left out for 
% lack of space. However this can be intuitively understood as follows:
% we can think of $\frac{\tilde{P}_{W_{\bar{i}}}(t)}{\tilde{P}(t)}$ as a weight
% that further reshapes the smoothed word-to-topic distribution $\widehat{P}_{w_i}(t)$
% to take into account the context. In our experiments we also smooth this distribution
% using {\em Laplacian} smoothing.
