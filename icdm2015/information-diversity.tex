The main objective of our work is to discover interestingness in text. However,
as we mentioned in Section~\ref{sec:introduction}, text interestingness problem is not well defined and has many different facets in different tasks/domains. In this work, we aim to demonstrate that in at least one important domain of e-commerce, text diversity can be used as a good proxy for 
predicting product interestingness. This does not mean that any diverse text is
expected to be interesting, but rather that a high diversity score in
a product is a good indicator of interestingness. However, we found
that none of the existing document diversity measures worked well in this setting. We believe
that the primary reason for this is that the amount of useful textual
data for each product is often too small to be treated as a document. 
In this section, we propose a new way of measuring diversity, that is
based on mutual information, and can be seen as a generalization of
Shannon entropy. This approach turns out to be effective even for
analyzing very short pieces of text. 


\subsection{Distributional Representations}
\label{sec:distributional-representations}
As described later in this section, the proposed diversity measure
is set in an information-theoretic framework. 
To apply it in the text setting we have to construct distributional
representations of each word in such a way that will preserve the
relevant content of the data.
This can be achieved through a common general model,
where we obtain a sizable collection of text with each word
occurrence tagged with one among a set of
contexts. 
% Moreover, naturally, every word occurrence can be mapped to a
% vocabulary term (barring ambiguities). 
So, each word in the vocabulary
corresponds to a group of its occurrences, which leads to a
distribution over the set of contexts. A distributional representation
over a vocabulary $V$ maps a word in the vocabulary to a  
probability distribution over a fixed set of contexts $T$  (for a brief review
see~\cite{Turian10wordrepresentations}). Often we start with a
co-occurrence matrix $\cM_{|V|\times|T|}$ where each row represents a
collection of co-occurrence counts of a word with the contexts. The
rows of $\cM_{|V|\times|T|}$ can then  be normalized in order to
obtain a probability distribution over the contexts.
For example, if we chose $T=V$ then we obtain the familiar {\sl
  word-to-word} co-occurrence representation which counts the number 
of times two words co-occurred in a document corpus. Another choice is
to use the set of topics learned over a document 
corpus by {\sl Latent Dirichlet Allocation}
(LDA)~\cite{Blei:2003:LDA:944919.944937}, as the contexts. In this case, each word is mapped to a probability distribution over the topics discovered by LDA. It has the
advantage of allowing us to select the appropriate dimensionality of
the representations by choosing the number of topics to learn. Since
this is the direction we take in this paper, from now on we will use
topic distributions as distributional representations of words.

%  We will use the notation $P_w$ to represent 
% the probability distribution over the context  given a word $w$. We
% also use the notation $W=\{w_1,...,w_k\}$ for a bag of word
% representation of a text snippet. Given a word distributional
% representation, $\cP_{W}=\{P_{w_ 1},...,P_{w_k}\}$ denotes the set
% of probability distributions over the context for the words 
% in the text snippet.

%%%%%%%%%%%%%%%%%%%%%%%%%%%%%%%%%%%%%%%%%%%%%%%%%%%%%%%%%%%%%

\subsection{Jensen-Shannon Diversity}
\label{sec:jensen-shannon-divergence}
% Why diversity?

% We pose the following question: what is the right model of diversity
% in text, with respect to the above framework? % Notice, however, that we
% need not restrict ourselves to text, because the way our
% distributional model is formulated can be applied to any set of
% labeled elements partitioned into groups (each group corresponding to
% a vocabulary word). Suppose we have a set $S$ of elements divided into
% a collection $G$ of groups, by a function $g:S\rightarrow
% G$. Additionally, each element has a label assigned with
% $f:S\rightarrow T$. What is the diversity between the groups?
% Our starting point is the notion of Shannon 
% entropy, generally considered as a well-founded measure of diversity 
% for a collection of elements represented by a
% distribution. Here, the collection consists of words, and each word
% appears many times in the data, so it provides
% finer information about the context assignments. Assume, for the sake
% of argument, that every occurrence of the same word had the same
% context assigned to it in a given corpus - in which case we could 
% uniquely identify a vocabulary term with a single context. We could then naturally obtain a
% context distribution for the vocabulary, weighed by the frequencies of
% each word in the dataset, and measure the entropy, thus obtaining the
% diversity of words in the vocabulary, with respect to the given corpus. Consider a word $w$
% sampled uniformly from the data. Let $X$ and $Y$ denote the random
% variables corresponding to the 
% appropriate vocabulary term and the assigned context,
% respectively. If, as we temporarily assumed,
% each vocabulary term has one uniquely assigned context, then $H(Y|X)=0$, so entropic diversity of the
% assignment, $H(Y)$, will be equal to the mutual information between $X$
% and $Y$: 
% \[I(X;Y) = H(Y) - H(Y|X) =H(Y).\]
% If we break the simplifying assumption, $H(Y)$ is no longer a
% reasonable metric, because if all of the words had contexts assigned
% uniformly at random, then their distributional representations would be
% identical, so there would be no diversity between them, even though $H(Y)$
% reached its maximum value. On the other hand, as we will see, mutual
% information $I(X;Y)$ provides a proper generalization of diversity in
% this case. Moreover, we can formulate mutual information in terms of the
% distributional representations of the words as follows:
%For the remainder of this section, we assume that the
%textual data has already been transformed into topic distributions.

For the remainder of this section, we assume that a topic distributional 
model for the words in the vocabulary has been learned. More specifically,  
we are given a distributional representation over a vocabulary $V$ and
topic set $T$, with $P_w$ giving a topic distribution corresponding to the
word $w$. We will measure information diversity
for a text snippet $W=\{w_1,...,w_k\}$ by analyzing its 
distributional representation
$\cP_{W}=\{P_{w_1},...,P_{w_k}\}$. Since we are ignoring the ordering
of words, this can be viewed as a bag-of-words model. To
estimate the diversity of $\cP_{W}$, we use a measure of divergence
between probability distributions, discussed in \cite{FugledeTopsoe},
which is a generalization of Jensen-Shannon Divergence.

\bed\label{diversity}
Let $\cP_{W}=\{P_{w_1},...,P_{w_k}\}$ be a set of
distributions (for example, a set of word topic distribution for the words in a sentence). Moreover, assume $\{d_{w_i}\}$  are a set of importance weights assigned to $\{P_{w_i}\}$, such that $\sum_{i=1}^kd_{w_i}=1$.
{\em Generalized Jensen-Shannon Divergence} of the mixture
$\Sigma d_{w_i}P_{w_i}$ is defined as
$$D_{JS}(\Sigma d_{w_i}P_{w_i})=\sum_{i=1}^k
d_{w_i}D_{KL}(P_{w_i}\|M),$$
where $M=\sum_{i=1}^kd_{w_i} P_{w_i}$ is the weighted average of the
distributions. As a shorthand, we will also write $D_{JS}(M)$, when
it is clear that $M$ represents a mixture of distributions.
\eed
% \bed\label{jsd-definition}
% Let $P(Y)=\Sigma d_w P_w$ be the distribution of contexts,
% described as a mixture of distributional word representations
% $P_w=P(Y\,|\,X\!=\!w)$, weighed by $d_w=P(X=w)$. Mutual information between
% $X$ and $Y$ can now be derived as
% \[ I(X;Y)= \sum_{w\in V} d_w D_{KL}(P_w\|P),\]
% where $D_{KL}$ is the Kullback-Leibler divergence.
% We will call it the {\bf Jensen-Shannon diversity} of that
% mixture, and denote by $D_{JS}(\Sigma d_wP_w)$.
% \eed

Here, $D_{KL}$ corresponds to Kullback-Leibler Divergence.
Note, that to compute the above quantity we need not only the
distributional representations, but also a set of weights assigned to
each of them. Intuitively, the weight $d_{w_i}$ should describe how
important the word $w_i$ is in describing the content of the text (for example, one may assign a higher importance  to the named entities in the text when measuring diversity, e.g., product related keywords in a product title). In
Section~\ref{sec:the-readers-model}, we will return to the issue of selecting the
right weights, since it is tied to the choice of distributions
themselves. The rest of this section is dedicated to the examination of the
basic properties of Generalized Jensen-Shannon Divergence (JSD) in the
context of measuring diversity. 

Interestingly, JSD can be defined as mutual information if we
interpret the object $\Sigma d_{w_i}P_{w_i}$ in a generative mixture
model setting. To that end, we define a multinomial distribution
$\mu$ over the words from $W$, based on the weights $d_{w_i}$. This
simple generative model first samples a word $w_\xi$ from $\mu$, then samples a
topic from the corresponding topic distribution $P_{w_\xi}$:
\begin{align*}
\xi &\sim \mu=\textrm{Mult}(d_{w_1},d_{w_2},\dots,d_{w_k}),\\
t &\sim P_{w_\xi}.
\end{align*}

We can now reformulate JSD in this model as the mutual information
between $\xi$ and $t$, through a straight-forward calculation
(see~\cite{Cover:2006:EIT:1146355} for an overview of mutual information).
\bep
Following the above notation, Generalized Jensen-Shannon Divergence
can be defined as
\[D_{JS}(\Sigma d_{w_i}P_{w_i})= I(\xi,t),\]
where $I(\xi,t)$ corresponds to the mutual information.
\eep

Based on this definition, JSD represents the average number of bits of
information learned about the topic of a word from knowing its
distributional representation. Also note, that mutual information is
symmetric with respect to $\xi$ and $t$, even though their roles in
our model are not. Let us examine some other basic properties of this
measure. 

\bep\label{jsd-properties}
Let $M=\Sigma d_{w_i} P_{w_i}$ be a mixture distribution. Jensen-Shannon
Divergence of $M$ satisfies the following properties:
 \begin{enumerate}
   \item $D_{JS}(\Sigma d_{w_i}P_{w_i})\geq 0$ with equality only when all
     distributions $P_{w_i}$ are equal.
   \item $D_{JS}(\Sigma d_{w_i}P_{w_i})\leq H(M)$ with equality when all
     $P_{w_i}$ have singleton support (i.e. $P(t)=1$ for some $t\in
     T$). Here, $H(M)$ is Shannon entropy.
 \end{enumerate}
\eep
The second property tells us that maximizing this measure means
pushing all distributions $P_w$ to the corners of the simplex and
therefore minimizing their uncertainty. From this, it follows that
Jensen-Shannon Divergence as a measure of diversity can be seen as a
direct generalization of Shannon entropy, when replacing fixed topic
assignments with distributional ones.  
