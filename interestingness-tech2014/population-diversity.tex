The observation in Proposition \ref{entropy} motivates us to look more closely into what type of
properties do we expect from a population diversity measure in the
case where an organism is assigned a probability distribution
over species. In the measure we propose above, if a distribution is
identical with the observer's prior distribution, then it gets zero
weight, and so has no impact on the diversity. This makes sense for
text, because we evaluate only the information it conveys. Think of
words, that play a mainly syntactic role (instead of semantic) in a
sentence. Our estimation of sentence diversity intuitively should not
depend on whether a given language has more syntactic words than
other. However, in
the case of a population of organisms, for example, we can think of
the distributions not as probabilities, but as proportions. In this
case, let us imagine a diverse population of 10 organisms, measured
against a uniform prior over species. If we were to add 90 organisms
with uniform distribution assigned to each, how should this change the
perceived diversity of the population? One way of looking at this is
that now 90\% of the organisms have identical characteristics, which
ought to indicate low diversity. We will now present an alteration of
our diversity measure that accomodates this intuition. For
consistency, let us keep the notation from the previous definitions,
although the word analogy may be less appropriate here. We will once
again try to encode the set $S=\{P_{w_1},...,P_{w_k}\}$, with respect
to a prior $P$. For simplicity, suppose that $\{w_1,...,w_k\}$ is a
binary code. Once again, we refer to the observation that encoding the
information in $P_{w_i}$ requires $D_{w_i}$, which means the
hypothetical corresponding code would contain $2^{D_{w_i}}$ words 
(ignoring the non-integrality of $D_{w_i}$). Denote the set of those
words as $V_{ij}=\{v_{i1},...,v_{il_i}\}$. As the complete code, then,
we will use $C=\{w_iv_{ij}|\ 1\leq i\leq k,\ 1\leq j\leq l_i\}$. Like
before, we can now describe an instantiation of the information source
model for the general Jensen-Shannon divergence. It will simply draw a
random word $u$ from $C$ uniformly, and then draw a topic from the
distribution $P_{w_i}$, where $w_i$ is a prefix of $u$. This
corresponds to using diversity as in Definition \ref{diversity}, but
with the weights  defined as
\[d'_{w_i}=\frac{2^{D_{w_i}}}{\sum_{j=1}^k2^{D_{w_j}}}.\]
We will call this the Jensen-Shannon Population Diversity. Notice,
that in this case all of the elements in the population will have
non-zero weights, no matter what their distribution is. This
definition also has some other nice properties when applied to
standard fixed-assignment populations. 

To present them, we will first
discuss a generative population model that incorporates the concept of
a prior species distribution. Suppose we want to measure the diversity
of a population $S$ within a larger universe $U$ of organisms assigned
species from the set $T$. We will assume that $S$ was generated from
$U$ as follows: We draw an element $u$ uniformly from $U$. Let $t\in
T$ be the species assignment for $u$. Next, we choose to add $u$ to
$S$ with probability $p_{S,t}$ (depending only on the species
assignment of $u$). For the diversity analysis, we will assume that
size of the population is large enough that it's species
distribution is has converged to the limit, and that the universe is
large enough (compared to the population) that sampling does not
noticeably affect its species distribution. We will now postulate
the following general properties that a diversity measure
should have in this model:
\begin{enumerate}
\item Given a fixed set of probabilities $p_{S,t}$ for the generative
  process, the population diversity does not depend on the species
  distribution of the universe.
\item If the species distribution of the universe is uniform, then we can
  revert to a standard model of population diversity. In our case, we
  will use Shannon entropy.
\end{enumerate}

\ber
A population that is uniformly sampled from the universe has
  highest diversity. 
\eer
This is a simple consequence of the two properties. In fact, we can
say something much more precise.

\bep
Let $S$ be a population sampled within universe $U$, where $P_S$ and
$P_U$ are the species distributions of $S$ and $U$,
respectively. Denote $T$ as the set of species. Let $Q$ be a species
distribution such that
\[Q(t)=\frac{P_S(t)(P_U(t))^{-1}}{\sum_{v\in T} P_S(v)(P_U(v))^{-1}}.\]
Then, the diversity of $S$ within $U$ is equal to $H(Q)$.
\eep

We can now come back to how this relates to the Jensen-Shannon
Population Diversity. Interestingly, it turns out that it is a direct
generalization of the generative population diversity model.

\bep
Let $S=\{P_{w_1},...,P_{w_k}\}$ be a set of distributions over species,
such that each has a singleton support. Let $P$ be the prior
species distribution corresponding to $S$. Then, the Jensen-Shannon
Population Diversity is equal to the diversity of $S$ as a population
within a universe with species distribution $P$.
\eep