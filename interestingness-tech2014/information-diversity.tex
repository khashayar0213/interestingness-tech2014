%
As a starting point of our model, we assume that we can describe any
word with a probability distribution over a fixed set of topics -
e.g. using a word-topic matrix of some topic model. Now, given a small
set of words, we in effect have a set of topic distributions. Let us
denote the set of words by $S=\{w_1,...,w_k\}$, and the distribution
of a word $w$ by $P_w$.  We now ask
what is the diversity of the set of probability
distributions $\cP_S=\{P_{w_1},...,P_{w_k}\}$? To make this task
somewhat more concrete, we make some 
assumptions of what we expect from the measure. First, if the set
$\cP_S$ only contains one element, then it is not diverse. Moreover,
we have a certain fixed {\em prior} distribution $P$, which we
can think of as the overall distribution of topics. If a distribution
$P_{w_i}$ is equal to $P$, then it does not carry any information, so
it should not have any impact on the diversity of $\cP_S$. In
accordance with those assumptions, we will will introduce some basic
notation. Let $D_{KL}(\centerdot\|\centerdot)$ denote the
Kullback-Leibler Divergence.

\bed
Given a distribution $P_w$, its {\bf importance} with respect to a
prior distribution $P$ is defined as 
\[D_w = D_{KL}(P_w\|P).\]
\eed

\bed\label{mixture}
Given a set of distributions $\cP_S=\{P_{w_1},...,P_{w_k}\}$ and a
prior $P$, we
define a mixture distribution $P_S$ as
\[P_S=\sum_{i=1}^k d_{w_i} P_{w_i},\]
where $d_{w_i}=\frac{D_{w_i}}{\sum D_{w_j}}$ are the normalized
importances.
\eed

Essentially, $P_S$ is the weighted average of the set $\cP_S$, where
the weights are chosen according to the importances. Now, we can
define the diversity measure.

\bed\label{diversity}
We define the Jensen-Shannon Information Diversity of a set of
distributions $\cP_S$ with respect to 
prior $P$ as
\[D_S=\sum_{i=1}^k d_{w_i}D_{KL}(P_{w_i}\|P_S), \]
where $d_{w_i}$ and $P_S$ are as in the previous definition.
\eed
This definition is closely related to the {\em general Jensen-Shannon
  Divergence}, defined in [topsoe]. Its information-theoretic interpretation
will help us understand why it can be an appropriate measure of
diversity. Let us consider an information source which selects one
topic from a fixed set, each selection being independent from the
previous ones. The selection is a two-step process:
\begin{enumerate}
\item first, one of the probability distributions is chosen from
  the set $\cP_S$, where $P_{w_i}$ is chosen with probability
  $d_{w_i}$;
\item then, the output topic is drawn from that distribution.
\end{enumerate}
Suppose, that we want to design an encoder $A_1$ for the topics, that is optimal
for this source (in terms of shortest average code length), assuming
that we know the mechanism of the source (including the $P_{w_i}$'s
and $d_{w_i}$'s), such that the encoder sees the source only as a black box
outputting the topics. Denote the shortest average code length achievable
by $A_1$ as $L_1$. Now, consider a second encoder $A_2$ with the same
goal, but this time it can look {\em into} the source and adjust its
encoding based on which distribution $P_{w_i}$ was selected by the
source (and we assume that a hypothetical decoder can also {\em
  look up} the word $w_i$). Denote the shortest average code length in this instance as
$L_2$. As discussed in [topsoe], the general Jensen-Shannon Divergence
$D_S$ is equal to $L_1-L_2$. In other words, it describes how much
information is contained in the selection of distribution $P_{w_i}$
within the information source. For example, in the boundary case where
all of the distributions in $\cP_S$ are identical, then $L_1=L_2$
because the initial selection is irrelevant and the set $\cP_S$ is not
diverse.  

Building on this interpretation, we will discuss a more detailed
thought experiment, which will argue for the choice of weights
$d_{w_i}$. Consider a set of words as a sentence that is supposed to
convey certain information. A word $w_i$ is described by the topic
distribution $P_{w_i}$, so the amount of information needed to
describe it is the divergence of $P_{w_i}$ from the Observer's prior
distribution $P$, which is precisely $D_{w_i}$. We can think of
$P_{w_i}$ as the distribution of Observer's thoughts, given that they
just saw the word $w_i$. So, if for example this distribution happens
to be equal to the prior, this means that the Observer's reaction is
independent from whether or not $w_i$ was read, in which case we
conclude, that there is no information there. 

To encode the information contained in the sentence, we can describe
each word using $D_{w_i}$ bits, and concatenate them together. We may
now ask what is the topic distribution for the Observer reading this
code bit by bit (or, say, randomly sampling bits from it), assuming
that at each step they use the distribution corresponding to the word
from which that bit comes from. Clearly, it will be the mixture
distribution $P_S$ from Definition \ref{mixture} and, moreover, what
we have just described is essentially an instance of the information
source model presented earlier. 

The diversity measure we are describing works in a different setting
than is typically assumed. The standard model for computing diversity
is derived from analysing a population of living organisms, each being
assigned a label from a fixed set of species. In our case, the species
are represented by topics and the organisms are represented by
words. The key difference, however, is that we do not force a fixed
assignment, but instead give a topic distribution for each
word. This can be reduced to a regular population model in the case where
every topic distribution has singleton support (set of topics with
non-zero probabilities). It is
reasonable to ask, then, how does our measure compare to other
population diversity measures. As it turns out, it is simply a
generalization of the Shannon entropy, which is one of the canonical
approaches to population diversity.

\bep\label{entropy}
Let $S=\{P_{w_1},...,P_{w_k}\}$ be a set of distributions over topics,
such that each has a singleton support. Assume that the prior
topic distribution $P$ corresponding to $S$ is a uniform
distribution. Then, 
\[D_S=H(P_S),\]
where $P_S$ corresponds to the overall topic distribution in the set
$S$, and $H$ denotes Shannon entropy.
\eep