\documentclass{beamer}

\mode<presentation>
{
  \usetheme{Warsaw}
  % or ...

  \setbeamercovered{transparent}
  % or whatever (possibly just delete it)
}

\usepackage[utf8]{inputenc}
\usepackage{amssymb}
\usepackage{amsmath}
\usepackage{subcaption}

%\DeclareMathOperator{\x}{\mathbf{x}}
\DeclareMathOperator{\e}{\mathbf{e}}
\DeclareMathOperator{\M}{\mathbf{M}}
\DeclareMathOperator{\w}{\mathbf{w}}
\newcommand{\fix}{\marginpar{FIX}}
\newcommand{\new}{\marginpar{NEW}}
\edef\polishl{\l} 



\RequirePackage{latexsym}
\RequirePackage{amsmath}
\RequirePackage{amssymb} 
\RequirePackage{color} 
\RequirePackage{bm}
\RequirePackage{color}
\RequirePackage{picinpar}

%%%%%%%% Stock standard definitions %%%%%%%%%%%%%%%

\newcommand{\wbt}{\widetilde{\mathbf{w}}}
\DeclareMathOperator{\ab}{\mathbf{a}}
\DeclareMathOperator{\abh}{\widehat{\ab}}
\DeclareMathOperator{\bb}{\mathbf{b}}
\DeclareMathOperator{\bbh}{\widehat{\bb}}
\DeclareMathOperator{\cb}{\mathbf{c}}
\DeclareMathOperator{\db}{\mathbf{d}}
\DeclareMathOperator{\eb}{\mathbf{e}}
\DeclareMathOperator{\fb}{\mathbf{f}}
\DeclareMathOperator{\gb}{\mathbf{g}}
\DeclareMathOperator{\hb}{\mathbf{h}}
\DeclareMathOperator{\ib}{\mathbf{i}}
\DeclareMathOperator{\jb}{\mathbf{j}}
\DeclareMathOperator{\kb}{\mathbf{k}}
\DeclareMathOperator{\lb}{\mathbf{l}}
\DeclareMathOperator{\mb}{\mathbf{m}}
\DeclareMathOperator{\nbb}{\mathbf{n}}
\DeclareMathOperator{\ob}{\mathbf{o}}
\DeclareMathOperator{\pb}{\mathbf{p}}
\DeclareMathOperator{\qb}{\mathbf{q}}
\DeclareMathOperator{\rb}{\mathbf{r}}
\DeclareMathOperator{\sbb}{\mathbf{s}}
\DeclareMathOperator{\tb}{\mathbf{t}}
\DeclareMathOperator{\ub}{\mathbf{u}}
\DeclareMathOperator{\vb}{\mathbf{v}}
\DeclareMathOperator{\wb}{\mathbf{w}}
\DeclareMathOperator{\xb}{\mathbf{x}}
\DeclareMathOperator{\yb}{\mathbf{y}}
\DeclareMathOperator{\zb}{\mathbf{z}}
\renewcommand{\l}{\ell}

\DeclareMathOperator{\atilde}{\tilde{\ab}}
\DeclareMathOperator{\btilde}{\tilde{\bb}}
\DeclareMathOperator{\ctilde}{\tilde{\cb}}
\DeclareMathOperator{\dtilde}{\tilde{\db}}
\DeclareMathOperator{\etilde}{\tilde{\eb}}
\DeclareMathOperator{\ftilde}{\tilde{\fb}}
\DeclareMathOperator{\gtilde}{\tilde{\gb}}
\DeclareMathOperator{\htilde}{\tilde{\hb}}
\DeclareMathOperator{\itilde}{\tilde{\ib}}
\DeclareMathOperator{\jtilde}{\tilde{\jb}}
\DeclareMathOperator{\ktilde}{\tilde{\kb}}
\DeclareMathOperator{\ltilde}{\tilde{\lb}}
\DeclareMathOperator{\mtilde}{\tilde{\mb}}
\DeclareMathOperator{\ntilde}{\tilde{\nbb}}
\DeclareMathOperator{\otilde}{\tilde{\ob}}
\DeclareMathOperator{\ptilde}{\tilde{\pb}}
\DeclareMathOperator{\qtilde}{\tilde{\qb}}
\DeclareMathOperator{\rtilde}{\tilde{\rb}}
\DeclareMathOperator{\stilde}{\tilde{\sbb}}
\DeclareMathOperator{\ttilde}{\tilde{\tb}}
\DeclareMathOperator{\utilde}{\tilde{\ub}}
\DeclareMathOperator{\vtilde}{\tilde{\vb}}
\DeclareMathOperator{\wtilde}{\tilde{\wb}}
\DeclareMathOperator{\xtilde}{\tilde{\xb}}
\DeclareMathOperator{\ytilde}{\tilde{\yb}}
\DeclareMathOperator{\ztilde}{\tilde{\zb}}

\DeclareMathOperator{\abar}{\bar{\ab}}
\DeclareMathOperator{\bbar}{\bar{\bb}}
\DeclareMathOperator{\cbar}{\bar{\cb}}
\DeclareMathOperator{\dbar}{\bar{\db}}
\DeclareMathOperator{\ebar}{\bar{\eb}}
\DeclareMathOperator{\fbar}{\bar{\fb}}
\DeclareMathOperator{\gbar}{\bar{\gb}}
\DeclareMathOperator{\hbbar}{\bar{\hb}}
\DeclareMathOperator{\ibar}{\bar{\ib}}
\DeclareMathOperator{\jbar}{\bar{\jb}}
\DeclareMathOperator{\kbar}{\bar{\kb}}
\DeclareMathOperator{\lbar}{\bar{\lb}}
\DeclareMathOperator{\mbar}{\bar{\mb}}
\DeclareMathOperator{\nbar}{\bar{\nbb}}
\DeclareMathOperator{\obar}{\bar{\ob}}
\DeclareMathOperator{\pbar}{\bar{\pb}}
\DeclareMathOperator{\qbar}{\bar{\qb}}
\DeclareMathOperator{\rbar}{\bar{\rb}}
\DeclareMathOperator{\sbar}{\bar{\sbb}}
\DeclareMathOperator{\tbar}{\bar{\tb}}
\DeclareMathOperator{\ubar}{\bar{\ub}}
\DeclareMathOperator{\vbar}{\bar{\vb}}
\DeclareMathOperator{\wbar}{\bar{\wb}}
\DeclareMathOperator{\xbar}{\bar{\xb}}
\DeclareMathOperator{\ybar}{\bar{\yb}}
\DeclareMathOperator{\zbar}{\bar{\zb}}

\DeclareMathOperator{\Ab}{\mathbf{A}}
\DeclareMathOperator{\Bb}{\mathbf{B}}
\DeclareMathOperator{\Cb}{\mathbf{C}}
\DeclareMathOperator{\Db}{\mathbf{D}}
\DeclareMathOperator{\Eb}{\mathbf{E}}
\DeclareMathOperator{\Fb}{\mathbf{F}}
\DeclareMathOperator{\Gb}{\mathbf{G}}
\DeclareMathOperator{\Hb}{\mathbf{H}}
\DeclareMathOperator{\Ib}{\mathbf{I}}
\DeclareMathOperator{\Jb}{\mathbf{J}}
\DeclareMathOperator{\Kb}{\mathbf{K}}
\DeclareMathOperator{\Lb}{\mathbf{L}}
\DeclareMathOperator{\Mb}{\mathbf{M}}
\DeclareMathOperator{\Nb}{\mathbf{N}}
\DeclareMathOperator{\Ob}{\mathbf{O}}
\DeclareMathOperator{\Pb}{\mathbf{P}}
\DeclareMathOperator{\Qb}{\mathbf{Q}}
\DeclareMathOperator{\Rb}{\mathbf{R}}
\DeclareMathOperator{\Sbb}{\mathbf{S}}
\DeclareMathOperator{\Tb}{\mathbf{T}}
\DeclareMathOperator{\Ub}{\mathbf{U}}
\DeclareMathOperator{\Vb}{\mathbf{V}}
\DeclareMathOperator{\Wb}{\mathbf{W}}
\DeclareMathOperator{\Xb}{\mathbf{X}}
\DeclareMathOperator{\Xbt}{\widetilde{\Xb}}
\DeclareMathOperator{\Xbh}{\widehat{\Xb}}
\DeclareMathOperator{\Xbs}{\widetilde{\Xb}}
\DeclareMathOperator{\Zbs}{\widetilde{\Zb}}
\DeclareMathOperator{\Kbs}{\widetilde{\Kb}}
\DeclareMathOperator{\Zbh}{\widehat{\Zb}}
\DeclareMathOperator{\Ubh}{\widehat{\Ub}}
\DeclareMathOperator{\Yb}{\mathbf{Y}}
\DeclareMathOperator{\Zb}{\mathbf{Z}}

\DeclareMathOperator{\Abar}{\bar{A}}
\DeclareMathOperator{\Bbar}{\bar{B}}
\DeclareMathOperator{\Cbar}{\bar{C}}
\DeclareMathOperator{\Dbar}{\bar{D}}
\DeclareMathOperator{\Ebar}{\bar{E}}
\DeclareMathOperator{\Fbar}{\bar{F}}
\DeclareMathOperator{\Gbar}{\bar{G}}
\DeclareMathOperator{\Hbar}{\bar{H}}
\DeclareMathOperator{\Ibar}{\bar{I}}
\DeclareMathOperator{\Jbar}{\bar{J}}
\DeclareMathOperator{\Kbar}{\bar{K}}
\DeclareMathOperator{\Lbar}{\bar{L}}
\DeclareMathOperator{\Mbar}{\bar{M}}
\DeclareMathOperator{\Nbar}{\bar{N}}
\DeclareMathOperator{\Obar}{\bar{O}}
\DeclareMathOperator{\Pbar}{\bar{P}}
\DeclareMathOperator{\Qbar}{\bar{Q}}
\DeclareMathOperator{\Rbar}{\bar{R}}
\DeclareMathOperator{\Sbar}{\bar{S}}
\DeclareMathOperator{\Tbar}{\bar{T}}
\DeclareMathOperator{\Ubar}{\bar{U}}
\DeclareMathOperator{\Vbar}{\bar{V}}
\DeclareMathOperator{\Wbar}{\bar{W}}
\DeclareMathOperator{\Xbar}{\bar{X}}
\DeclareMathOperator{\Ybar}{\bar{Y}}
\DeclareMathOperator{\Zbar}{\bar{Z}}

\DeclareMathOperator{\Abbar}{\bar{\Ab}}
\DeclareMathOperator{\Bbbar}{\bar{\Bb}}
\DeclareMathOperator{\Cbbar}{\bar{\Cb}}
\DeclareMathOperator{\Dbbar}{\bar{\Db}}
\DeclareMathOperator{\Ebbar}{\bar{\Eb}}
\DeclareMathOperator{\Fbbar}{\bar{\Fb}}
\DeclareMathOperator{\Gbbar}{\bar{\Gb}}
\DeclareMathOperator{\Hbbar}{\bar{\Hb}}
\DeclareMathOperator{\Ibbar}{\bar{\Ib}}
\DeclareMathOperator{\Jbbar}{\bar{\Jb}}
\DeclareMathOperator{\Kbbar}{\bar{\Kb}}
\DeclareMathOperator{\Lbbar}{\bar{\Lb}}
\DeclareMathOperator{\Mbbar}{\bar{\Mb}}
\DeclareMathOperator{\Nbbar}{\bar{\Nb}}
\DeclareMathOperator{\Obbar}{\bar{\Ob}}
\DeclareMathOperator{\Pbbar}{\bar{\Pb}}
\DeclareMathOperator{\Qbbar}{\bar{\Qb}}
\DeclareMathOperator{\Rbbar}{\bar{\Rb}}
\DeclareMathOperator{\Sbbar}{\bar{\Sb}}
\DeclareMathOperator{\Tbbar}{\bar{\Tb}}
\DeclareMathOperator{\Ubbar}{\bar{\Ub}}
\DeclareMathOperator{\Vbbar}{\bar{\Vb}}
\DeclareMathOperator{\Wbbar}{\bar{\Wb}}
\DeclareMathOperator{\Xbbar}{\bar{\Xb}}
\DeclareMathOperator{\Ybbar}{\bar{\Yb}}
\DeclareMathOperator{\Zbbar}{\bar{\Zb}}

\DeclareMathOperator{\Ahat}{\widehat{A}}
\DeclareMathOperator{\Bhat}{\widehat{B}}
\DeclareMathOperator{\Chat}{\widehat{C}}
\DeclareMathOperator{\Dhat}{\widehat{D}}
\DeclareMathOperator{\Ehat}{\widehat{E}}
\DeclareMathOperator{\Fhat}{\widehat{F}}
\DeclareMathOperator{\Ghat}{\widehat{G}}
\DeclareMathOperator{\Hhat}{\widehat{H}}
\DeclareMathOperator{\Ihat}{\widehat{I}}
\DeclareMathOperator{\Jhat}{\widehat{J}}
\DeclareMathOperator{\Khat}{\widehat{K}}
\DeclareMathOperator{\Lhat}{\widehat{L}}
\DeclareMathOperator{\Mhat}{\widehat{M}}
\DeclareMathOperator{\Nhat}{\widehat{N}}
\DeclareMathOperator{\Ohat}{\widehat{O}}
\DeclareMathOperator{\Phat}{\widehat{P}}
\DeclareMathOperator{\Qhat}{\widehat{Q}}
\DeclareMathOperator{\Rhat}{\widehat{R}}
\DeclareMathOperator{\Shat}{\widehat{S}}
\DeclareMathOperator{\That}{\widehat{T}}
\DeclareMathOperator{\Uhat}{\widehat{U}}
\DeclareMathOperator{\Vhat}{\widehat{V}}
\DeclareMathOperator{\What}{\widehat{W}}
\DeclareMathOperator{\Xhat}{\widehat{X}}
\DeclareMathOperator{\Yhat}{\widehat{Y}}
\DeclareMathOperator{\Zhat}{\widehat{Z}}

\DeclareMathOperator{\Abhat}{\widehat{\Ab}}
\DeclareMathOperator{\Bbhat}{\widehat{\Bb}}
\DeclareMathOperator{\Cbhat}{\widehat{\Cb}}
\DeclareMathOperator{\Dbhat}{\widehat{\Db}}
\DeclareMathOperator{\Ebhat}{\widehat{\Eb}}
\DeclareMathOperator{\Fbhat}{\widehat{\Fb}}
\DeclareMathOperator{\Gbhat}{\widehat{\Gb}}
\DeclareMathOperator{\Hbhat}{\widehat{\Hb}}
\DeclareMathOperator{\Ibhat}{\widehat{\Ib}}
\DeclareMathOperator{\Jbhat}{\widehat{\Jb}}
\DeclareMathOperator{\Kbhat}{\widehat{\Kb}}
\DeclareMathOperator{\Lbhat}{\widehat{\Lb}}
\DeclareMathOperator{\Mbhat}{\widehat{\Mb}}
\DeclareMathOperator{\Nbhat}{\widehat{\Nb}}
\DeclareMathOperator{\Obhat}{\widehat{\Ob}}
\DeclareMathOperator{\Pbhat}{\widehat{\Pb}}
\DeclareMathOperator{\Qbhat}{\widehat{\Qb}}
\DeclareMathOperator{\Rbhat}{\widehat{\Rb}}
\DeclareMathOperator{\Sbhat}{\widehat{\Sb}}
\DeclareMathOperator{\Tbhat}{\widehat{\Tb}}
\DeclareMathOperator{\Ubhat}{\widehat{\Ub}}
\DeclareMathOperator{\Vbhat}{\widehat{\Vb}}
\DeclareMathOperator{\Wbhat}{\widehat{\Wb}}
\DeclareMathOperator{\Xbhat}{\widehat{\Xb}}
\DeclareMathOperator{\Ybhat}{\widehat{\Yb}}
\DeclareMathOperator{\Zbhat}{\widehat{\Zb}}

\DeclareMathOperator{\Acal}{\mathcal{A}}
\DeclareMathOperator{\Bcal}{\mathcal{B}}
\DeclareMathOperator{\Ccal}{\mathcal{C}}
\DeclareMathOperator{\Dcal}{\mathcal{D}}
\DeclareMathOperator{\Ecal}{\mathcal{E}}
\DeclareMathOperator{\Fcal}{\mathcal{F}}
\DeclareMathOperator{\Gcal}{\mathcal{G}}
\DeclareMathOperator{\Hcal}{\mathcal{H}}
\DeclareMathOperator{\Ical}{\mathcal{I}}
\DeclareMathOperator{\Jcal}{\mathcal{J}}
\DeclareMathOperator{\Kcal}{\mathcal{K}}
\DeclareMathOperator{\Lcal}{\mathcal{L}}
\DeclareMathOperator{\Mcal}{\mathcal{M}}
\DeclareMathOperator{\Ncal}{\mathcal{N}}
\DeclareMathOperator{\Ocal}{\mathcal{O}}
\DeclareMathOperator{\Pcal}{\mathcal{P}}
\DeclareMathOperator{\Qcal}{\mathcal{Q}}
\DeclareMathOperator{\Rcal}{\mathcal{R}}
\DeclareMathOperator{\Scal}{\mathcal{S}}
\DeclareMathOperator{\Scalt}{\widetilde{\Scal}}
\DeclareMathOperator{\Tcal}{\mathcal{T}}
\DeclareMathOperator{\Ucal}{\mathcal{U}}
\DeclareMathOperator{\Vcal}{\mathcal{V}}
\DeclareMathOperator{\Wcal}{\mathcal{W}}
\DeclareMathOperator{\Xcal}{\mathcal{X}}
\DeclareMathOperator{\Ycal}{\mathcal{Y}}
\DeclareMathOperator{\Zcal}{\mathcal{Z}}

\DeclareMathOperator{\Atilde}{\widetilde{A}}
\DeclareMathOperator{\Btilde}{\widetilde{B}}
\DeclareMathOperator{\Ctilde}{\widetilde{C}}
\DeclareMathOperator{\Dtilde}{\widetilde{D}}
\DeclareMathOperator{\Etilde}{\widetilde{E}}
\DeclareMathOperator{\Ftilde}{\widetilde{F}}
\DeclareMathOperator{\Gtilde}{\widetilde{G}}
\DeclareMathOperator{\Htilde}{\widetilde{H}}
\DeclareMathOperator{\Itilde}{\widetilde{I}}
\DeclareMathOperator{\Jtilde}{\widetilde{J}}
\DeclareMathOperator{\Ktilde}{\widetilde{K}}
\DeclareMathOperator{\Ltilde}{\widetilde{L}}
\DeclareMathOperator{\Mtilde}{\widetilde{M}}
\DeclareMathOperator{\Ntilde}{\widetilde{N}}
\DeclareMathOperator{\Otilde}{\widetilde{O}}
\DeclareMathOperator{\Ptilde}{\widetilde{P}}
\DeclareMathOperator{\Qtilde}{\widetilde{Q}}
\DeclareMathOperator{\Rtilde}{\widetilde{R}}
\DeclareMathOperator{\Stilde}{\widetilde{S}}
\DeclareMathOperator{\Ttilde}{\widetilde{T}}
\DeclareMathOperator{\Utilde}{\widetilde{U}}
\DeclareMathOperator{\Vtilde}{\widetilde{V}}
\DeclareMathOperator{\Wtilde}{\widetilde{W}}
\DeclareMathOperator{\Xtilde}{\widetilde{X}}
\DeclareMathOperator{\Ytilde}{\widetilde{Y}}
\DeclareMathOperator{\Ztilde}{\widetilde{Z}}


%%%%%%%% Widely accepted definitions %%%%%%%%%%%%%%%

\DeclareMathOperator{\CC}{\mathbb{C}} % Complex numbers
\DeclareMathOperator{\EE}{\mathbb{E}} % Expectation
\DeclareMathOperator{\KK}{\mathbb{K}} % Arbitrary field
\DeclareMathOperator{\MM}{\mathbb{M}} % Median
\DeclareMathOperator{\NN}{\mathbb{N}} % Natural numbers
\DeclareMathOperator{\PP}{\mathbb{P}} % Probability
\DeclareMathOperator{\QQ}{\mathbb{Q}} % Rationals
\DeclareMathOperator{\RR}{\mathbb{R}} % Real numbers 
\DeclareMathOperator{\ZZ}{\mathbb{Z}} % Integers

\DeclareMathOperator{\one}{\mathbf{1}}  % Identity
\DeclareMathOperator{\zero}{\mathbf{0}} % Zero
\DeclareMathOperator{\TRUE}{\mathbf{TRUE}}  % True
\DeclareMathOperator{\FALSE}{\mathbf{FALSE}}  % False

\DeclareMathOperator*{\mini}{\mathop{\mathrm{minimize}}}
\DeclareMathOperator*{\maxi}{\mathop{\mathrm{maximize}}}
\DeclareMathOperator*{\argmin}{\mathop{\mathrm{argmin}}}
\DeclareMathOperator*{\argmax}{\mathop{\mathrm{argmax}}}
\DeclareMathOperator*{\argsup}{\mathop{\mathrm{argsup}}}
\DeclareMathOperator*{\arginf}{\mathop{\mathrm{arginf}}}
\DeclareMathOperator{\sgn}{\mathop{\mathrm{sign}}}
\DeclareMathOperator{\sign}{\mathop{\mathrm{sign}}}
\DeclareMathOperator{\tr}{\mathop{\mathrm{tr}}}
\DeclareMathOperator{\rank}{\mathop{\mathrm{rank}}}
\DeclareMathOperator{\traj}{\mathop{\mathrm{Traj}}}

%%%%%%%% Bold Greek Letters %%%%%%%%%%%%%%%
\DeclareMathOperator{\sigmab}{\bm{\sigma}}
\DeclareMathOperator{\Sigmab}{\mathbf{\Sigma}}


%%%%%%%% Mess around with LaTeX %%%%%%%%%%%%%%%

%% Some style files might actually define these variables.
%% So don't mess with them if they are already defined

\ifx\BlackBox\undefined
\newcommand{\BlackBox}{\rule{1.5ex}{1.5ex}}  % end of proof
\fi

\ifx\proof\undefined
\newenvironment{proof}{\par\noindent{\bf Proof\ }}{\hfill\BlackBox\\[2mm]}
% \else
% \renewenvironment{proof}{\par\noindent{\bf Proof\ }}{\hfill\BlackBox\\[2mm]}
\fi

%Trying to put all on Section track
%\makeatletter
%\@addtoreset{equation}{section}
%\def\theequation{\thesection.\arabic{equation}}
%\def\thetheorem{\thesection.\arabic{theorem}}
%\makeatother

%the below clashes with the previous defs of them... 
%in the style files
%\newtheorem{theorem}{Theorem}[section]
%\newtheorem{lemma}[theorem]{Lemma}
%\newtheorem{corollary}[theorem]{Corollary}
%\newtheorem{conjecture}[conjecture]{Conjecture}


%%%%%%%% Utility functions %%%%%%%%%%%%%%%

\newcommand{\eq}[1]{(\ref{#1})} 
\newcommand{\mymatrix}[2]{\left[\begin{array}{#1} #2 \end{array}\right]}
\newcommand{\mychoose}[2]{\left(\begin{array}{c} #1 \\ #2 \end{array}\right)}
\newcommand{\mydet}[1]{\det\left[ #1 \right]}
\newcommand{\sembrack}[1]{[\![#1]\!]}

\newcommand{\ea}{\emph{et al. }}
\newcommand{\eg}{\emph{e.g. }}
\newcommand{\ie}{\emph{i.e., }}

\newcommand{\mnote}[1]{\marginpar{#1}}
\newcommand{\note}[1]{{\bf {#1}}}

%%%%%%%% Specific symbols for this project %%%%%%%%%%%%%%%

\DeclareMathOperator{\half}{\frac{1}{2}}

\newcommand{\Ref}[1]{\hfill\Green{[#1]}}
\DeclareMathOperator{\XX}{\mathcal{X}}
\newcommand{\ar}{\implies}
\newcommand{\yh}{\hat{y}}
\DeclareMathOperator{\lan}{\langle}
\DeclareMathOperator{\ran}{\rangle}

\DeclareMathOperator{\dc}{\mathrm{dc}}

\DeclareMathOperator{\Utb}{\widetilde{\Ub}}
\DeclareMathOperator{\Stb}{\widetilde{\Sbb}}

\ifx\Brown\undefined
\definecolor{brown}{rgb}{0.5,0.1,0.1}
\newcommand{\Brown}[1]{\color{brown}{#1}\color{black}}
\fi

\ifx\Red\undefined
\definecolor{red}{rgb}{1.0,0,0}
\newcommand{\Red}[1]{\color{red}{#1}\color{black}}
\fi

\ifx\Green\undefined
\definecolor{green}{rgb}{0,0.4,0}
\newcommand{\Green}[1]{\color{green}{#1}\color{black}}
\fi

\ifx\Blue\undefined
\definecolor{blue}{rgb}{0,0,1.0}
\newcommand{\Blue}[1]{\color{blue}{#1}\color{black}}
\fi



\newcommand{\beq}{\begin{equation}}
\newcommand{\eeq}{\end{equation}}
\newcommand{\beh}{\begin{conjecture}}
\newcommand{\eeh}{\end{conjecture}}
\newcommand\bel{\begin{lemma}}
\newcommand\eel{\end{lemma}}
\newcommand\bet{\begin{theoreme}}
\newcommand\eet{\end{theoreme}}
\newcommand\bex{\begin{example}}
\newcommand\eex{\end{example}}
\newcommand\bed{\begin{definition}}
\newcommand\eed{\end{definition}}
\newcommand\bep{\begin{proposition}}
\newcommand\eep{\end{proposition}}
\newcommand\ber{\begin{remark}}
\newcommand\eer{\end{remark}}
\newcommand\bec{\begin{corollary}}
\newcommand\eec{\end{corollary}}
%\newcommand\proof{\noindent {\bf Proof.}\ \ }
\newcommand\qed{\hfill$\Box$\medskip}
\newcommand\cH{{\mathcal H}}
\newcommand\cP{{\mathcal P}}
\newcommand\cJ{{\mathcal J}}
\newcommand\cD{{\mathcal D}}
\newcommand\cC{{\mathcal C}}
\newcommand\cO{{\mathcal O}}
\newcommand\cS{{\mathcal S}}
\newcommand\cT{{\mathcal T}}
\newcommand\cV{{\mathcal V}}
\newcommand\cW{{\mathcal W}}
\newcommand\cY{{\mathcal Y}}
\newcommand\cF{{\mathcal F}}
\newcommand\cU{{\mathcal U}}
\newcommand\cE{{\mathcal E}}
\newcommand\cG{{\mathcal G}}
\newcommand\cB{{\mathcal B}}
\newcommand\cI{{\rm I}}
\newcommand\cN{{\mathcal N}}
\newcommand\cM{{\mathcal M}}
\newcommand\cA{{\mathcal A}}
\newcommand\cQ{{\mathcal Q}}
\newcommand\cK{{\mathcal K}}
\newcommand\cZ{{\mathcal Z}}
\newcommand\CAP{{\rm cap}}
\newcommand\ENT{{\rm ent}}
\newcommand\gr{{\rm gr}}

\def\qq{{\mathbb Q}}
\def\ff{{\mathbb F}}
\def\rr{{\mathbb R}}
\def\zz{{\mathbb Z}}
\def\cc{{\mathbb C}}
\def\nn{{\mathbb N}}
\def\kk{{\mathbb K}}
\def\ee{{\mathbb E}}
\def\ww{{\mathbb W}}
\def\hh{{\mathbb H}}
\def\ss{{\mathbb S}}
\def\tt{{\mathbb T}}
\def\pp{{\mathbb P}}



% \newtheorem{theoreme}{Theorem} %[section]
% \newtheorem{proposition}[theoreme]{Proposition}
% \newtheorem{lemma}[theoreme]{Lemma}
% \newtheorem{definition}[theoreme]{Definition}
% \newtheorem{corollary}[theoreme]{Corollary}
% \newtheorem{remark}[theoreme]{Remark}
% \newtheorem{example}[theoreme]{Example}
% \newtheorem{examples}[theoreme]{Examples}
% \newtheorem{conjecture}[theoreme]{Conjecture}
% \newtheorem{conjecture}{Conjecture}
\newcommand{\beq}{\begin{equation}}
\newcommand{\eeq}{\end{equation}}
\newcommand{\beh}{\begin{Conjecture}}
\newcommand{\eeh}{\end{Conjecture}}
\newcommand\bel{\begin{Lemma}}
\newcommand\eel{\end{Lemma}}
\newcommand\bet{\begin{Theorem}}
\newcommand\eet{\end{Theorem}}
\newcommand\bex{\begin{Example}}
\newcommand\eex{\end{Example}}
\newcommand\bed{\begin{Definition}}
\newcommand\eed{\end{Definition}}
\newcommand\bep{\begin{Proposition}}
\newcommand\eep{\end{Proposition}}
\newcommand\ber{\begin{Remark}}
\newcommand\eer{\end{Remark}}
\newcommand\bec{\begin{Corollary}}
\newcommand\eec{\end{Corollary}}
%\newcommand\proof{\noindent {\bf Proof.}\ \ }
%\newcommand\qed{\hfill$\Box$\medskip}
\newcommand\cP{{\mathcal P}}
\newcommand\cJ{{\mathcal J}}
\newcommand\cD{{\mathcal D}}
\newcommand\cC{{\mathcal C}}
\newcommand\cO{{\mathcal O}}
\newcommand\cS{{\mathcal S}}
\newcommand\cT{{\mathcal T}}
\newcommand\cV{{\mathcal V}}
\newcommand\cW{{\mathcal W}}
\newcommand\cY{{\mathcal Y}}
\newcommand\cF{{\mathcal F}}
\newcommand\cU{{\mathcal U}}
\newcommand\cE{{\mathcal E}}
\newcommand\cG{{\mathcal G}}
\newcommand\cB{{\mathcal B}}
\newcommand\cI{{\rm I}}
\newcommand\cN{{\mathcal N}}
\newcommand\cM{{\mathcal M}}
\newcommand\cA{{\mathcal A}}
\newcommand\cQ{{\mathcal Q}}
\newcommand\cK{{\mathcal K}}
\newcommand\cZ{{\mathcal Z}}
\newcommand\CAP{{\rm cap}}
\newcommand\ENT{{\rm ent}}
\newcommand\gr{{\rm gr}}

\def\qq{{\mathbb Q}}
\def\ff{{\mathbb F}}
\def\rr{{\mathbb R}}
\def\zz{{\mathbb Z}}
\def\cc{{\mathbb C}}
\def\nn{{\mathbb N}}
\def\kk{{\mathbb K}}
\def\ee{{\mathbb E}}
\def\ww{{\mathbb W}}
\def\hh{{\mathbb H}}
\def\ss{{\mathbb S}}
\def\tt{{\mathbb T}}
\def\pp{{\mathbb P}}


\setkeys{Gin}{width=0.7\textwidth}

\title[Model of Text Interestingness]{An Information Theoretic
    Model of Quantifying Text Interestingness}  


\author{Michał Dereziński}

\begin{document}

\begin{frame}
  \titlepage
\end{frame}

\linespread{1.3}

\section{Introduction}
\subsection{Problem Statement}
\begin{frame}
\frametitle{Finding Interesting Products}
\begin{itemize}
\item {\bf Goal:} Increase user engagement when shopping online.
\item {\bf Solution:} Display ads for products that are {\em unique},
  {\em surprising}, {\em interesting}. 
\item Users are unlikely to buy such products, but they will browse
  them and click on the ads. 
\end{itemize}
\end{frame}

\begin{frame}
\frametitle{Example: Random Phone Cases}
\begin{figure}
\centering
\makebox[\textwidth][c]{\includegraphics[width=\textwidth]{figures/boring.png}}
\label{fig:boring}
\end{figure}
\end{frame}

\begin{frame}
\frametitle{Example: Interesting Phone Cases}
\begin{figure}
\centering
\makebox[\textwidth][c]{\includegraphics[width=\textwidth]{figures/interesting.png}}
\label{fig:interesting}
\end{figure}
\end{frame}

\subsection{Hypothesis}

\begin{frame}
\frametitle{Interestingness as Text Diversity}
\begin{itemize}
\item We used eBay listing titles as the features.
\item Our hypothesis is that often an interesting concept is generated
  by mixing a diverse set of topics.
\item We can identify those topics from the words appearing in the
  product title (usually 10-12 words).
\end{itemize}
\end{frame}

\begin{frame}
\frametitle{Example: Diversity in Product Titles}
\begin{figure}
\label{fig:ebay-products}
        \centering
         \begin{subfigure}[b]{0.3\textwidth}
        	        \centering
                \includegraphics[width=25mm]{figures/eyeshadow-iphone-case.jpg}
                \caption{\textcolor{red}{{\bf Eyeshadow}} Palettes for \textcolor{red}{{\bf iPhone}} 6 case}
                \label{fig:eyeshadow-iphone-case}
        \end{subfigure}
              ~ %add desired spacing between images, e. g. ~, \quad, \qquad, \hfill etc.
          %(or a blank line to force the subfigure onto a new line)
        \begin{subfigure}[b]{0.3\textwidth}
		 \centering
                \includegraphics[width=25mm]{figures/horn-iphone-speaker.jpg}
\caption{White Silicone \textcolor{red}{{\bf Horn}} Stand Speaker for Apple \textcolor{red}{{\bf iPhone}} 4/ 4S}                \label{fig:zeppelin-speaker}
        \end{subfigure}
       ~ %add desired spacing between images, e. g. ~, \quad, \qquad, \hfill etc.
          %(or a blank line to force the subfigure onto a new line)
        \begin{subfigure}[b]{0.3\textwidth}
		 \centering
                \includegraphics[width=30mm]{figures/geeky-clock.jpg}
\caption{\textcolor{red}{{\bf Equation}} Wall \textcolor{red}{{\bf
      Clock}} Gifts for Math Gurus}                
\label{fig:geeky-clock}
        \end{subfigure}
       % \caption{A collection of unique/interesting {\em eBay}
       % products. Highlighted keywords demonstrate how the text
       % associated with such products could span multiple diverse
       % topics.}
\end{figure}
\end{frame}

\subsection{Model Assumptions}

\begin{frame}
\frametitle{Distributional Word Representation}
\begin{itemize}
\item Each word in the vocabulary $V$ is mapped to a probability
  distribution over a fixed domain $T$.
\item Often, we start with a {\bf co-occurrence matrix} $\cM_{|V|\times|T|}$
  where each row represents a co-occurrence of a word with the set
  $T$. 
% \item For $T=V$, we obtain the familiar {\em word-to-word}
%   co-occurrence representation.
\item We chose to use {\bf topic modeling} where $T$ is the set of
  topics learned over a document corpus by {\em Latent Dirichlet
    Allocation}. 
\end{itemize}
\end{frame}

\begin{frame}
\frametitle{Co-occurrence Matrix}
\begin{center}
\begin{tabular}{c|ccc}
%\rowcolor{yellow} \cellcolor{lightyellow}
$\cM$& {\it Phones} & {\it Computers} & {\it Movies}\\
\hline
%\rowcolor{lightestyellow} \cellcolor{yellow} 
{\bf Smartphone} & 111 & 3 & 1\\
%\rowcolor{blue}% \cellcolor{blue}
 {\bf Screen} & 60 & 40 & 0\\
 {\bf Mouse} & 0 & 30 & 3\\
%\rowcolor{lightestyellow} \cellcolor{yellow}
 {\bf Batman} & 7 & 0 & 65\\
%\rowcolor{lightestyellow} \cellcolor{yellow} 
{\bf New} & 325 & 240 & 165\\
%\rowcolor{lightestyellow} \cellcolor{yellow} 
{\bf Esoteric} & 0 & 0 & 1
\end{tabular}
\end{center}
\begin{itemize}
\item Each row gives a probability distribution of topic 
assignments. 
\item But, are those good representations for words?
\end{itemize}
\end{frame}

\begin{frame}
\frametitle{The Reader's Model}
\begin{itemize}
\item Consider a person reading a text $W=\{w_1,...,w_k\}$. 
\item The Reader represents meaning of each word $w_i$ with a probability
  distribution $P_{i}$ over the set of topics $T$, e.g.:
$$\textrm{Screen} \rightarrow
\left\{\overset{\textrm{Phones}}{0.6},\overset{\textrm{Computers}}{0.4},
\overset{\textrm{Movies}}{0}\right\}.$$
% \item The Reader's {\em knowledge} is matrix $\cM_{|V|\times|T|}$ coming
%   from an LDA topic model trained on a separate corpus of documents.
\item We will address two questions:
\begin{enumerate}
\item How should the Reader choose $P_{i}$, given $W$ and
  $\cM_{|V|\times|T|}$?
\item Given $\cP_{W}=\{P_{1},...,P_{k}\}$, knowing
  $\cM_{|V|\times|T|}$, how should the Reader measure diversity of $W$?
\end{enumerate}
\end{itemize}
\end{frame}

\section{The Reader's Model}

\subsection{Information Diversity}

\begin{frame}
\frametitle{Jensen-Shannon Divergence}
\bed
The classical Jensen-Shannon Divergence between distributions $P$ and
$Q$ is defined (denoting $M=\frac{1}{2}(P+Q)$) as
\[D_{JS}(P,Q) = \frac{1}{2}\left(D_{KL}(P\|M)+D_{KL}(Q\|M)\right).\]
\eed
\bed
Generalized Jensen-Shannon Divergence for  $\cP=\{P_{1},...,P_{k}\}$
and weights $d_1+...+d_k=1$ is defined (denoting $M=\sum_{i=1}^k d_i
P_i$) as
\[D_{JS}(\Sigma d_iP_i) = \sum_{i=1}^k d_i D_{KL}(P_i\|M).\] 
\eed
\end{frame}

\begin{frame}
\frametitle{Information Diversity}
\bed\label{importance}
Given a distribution $P_i$, its {\sl importance} with respect to a
prior distribution $P$ is defined as $D_i = D_{KL}(P_i\|P)$.
\eed
\begin{itemize}
\item Here, $P_i$ represents a topic distribution for some word
  $w_i$. 
\item We can write it as a conditional distribution
  $\pp(t|w_i,\cB)$, where $\cB$ represents background knowledge. 
\item In this context, the prior $P$ would correspond to
  $\pp(t|\cB)$. 
\item {\em Importance} represents the amount of information obtained
  by conditioning on $w_i$.
\end{itemize}
\end{frame}

\begin{frame}
\frametitle{Information Diversity, cont.}
\bed\label{diversity}
We define the Jensen-Shannon Information Diversity of $\cP$ with
respect to prior $P$ as $\mbox{JSID}_P(\cP)=D_{JS}(\Sigma d_i P_i)$
where $d_{i}=\frac{D_{i}}{\sum D_{j}}$ are the normalized importances.
\eed
\bet\label{entropy}
Let $\cP$ be such that each $P_i$ has a singleton support. Assume that
the prior $U$ is in this case uniform. Then, 
$JSID_U(\cP)=H(P_S)$,
where $P_S$ corresponds to the overall distribution in the set
$S$, and $H$ denotes Shannon entropy.
\eet
\end{frame}

\begin{frame}
\frametitle{Information Diversity: Example}
\begin{figure}
\label{fig:diversity}
        \centering
         \begin{subfigure}[b]{0.45\textwidth}
        	        \centering
                \includegraphics[width=50mm]{figures/diversity1.jpg}
                \caption{Low diversity}
        \end{subfigure}
              ~ %add desired spacing between images, e. g. ~, \quad, \qquad, \hfill etc.
          %(or a blank line to force the subfigure onto a new line)
        \begin{subfigure}[b]{0.45\textwidth}
		 \centering
                \includegraphics[width=50mm]{figures/diversity2.jpg}
                \caption{{\bf High} diversity}
        \end{subfigure}
       ~ %add desired spacing between images, e. g. ~, \quad, \qquad, \hfill etc.
          %(or a blank line to force the subfigure onto a new line)
        \begin{subfigure}[b]{0.45\textwidth}
		 \centering
                \includegraphics[width=50mm]{figures/diversity3.jpg}
\caption{Low diversity}
        \end{subfigure}
       ~ %add desired spacing between images, e. g. ~, \quad, \qquad, \hfill etc.
          %(or a blank line to force the subfigure onto a new line)
        \begin{subfigure}[b]{0.45\textwidth}
		 \centering
                \includegraphics[width=50mm]{figures/diversity4.jpg}
\caption{Low diversity}
        \end{subfigure}
\end{figure}
\end{frame}

% \subsection{Population Diversity (digression)}

% \begin{frame}
% \frametitle{Population Diversity (digression)}
% \begin{itemize}
% \item Information Diversity works well with the bag of words
%   interpretation. 
% \item However, its relation with entropy suggests a
%   connection with population diversity.
% \item What is the role of the prior distribution in a population?
% \item Why is the prior uniform in the theorem? What if it is not?
% \end{itemize}
% \end{frame}

% \begin{frame}
% \frametitle{Generative Population Model}
% \begin{itemize}
% \item Consider a population $S$ of
%   organisms assigned species from the set $T$. $S$ was generated from a universe
%   $U$ by:
% \begin{enumerate}
% \item We draw an element $u$ uniformly from $U$. Let $t\in
% T$ be the species assignment for $u$.
% \item We choose to add $u$ to
% $S$ with probability $p_{S,t}$ (depending only on the species
% assignment of $u$).
% \end{enumerate}
% \item We will assume that:
% \begin{enumerate}
% \item $S$ is large enough that its species
%   distribution has {\em converged}.
% \item $U$ is large enough compared to $S$, so that sampling from it
%   does not noticeably affect its species distribution.
% \end{enumerate}
% \end{itemize}
% \end{frame}

% \begin{frame}
% \frametitle{Axiomatic Definition of Diversity}
% We will now postulate
% the following general properties that a diversity measure
% should have in this model:
% \begin{enumerate}
% \item Given a fixed set of parameters $p_{S,t}$ for the generative
%   process, the population diversity does not depend on the species
%   distribution of the universe.
% \item If the species distribution of the universe is uniform, then we can
%   revert to a standard model of population diversity. In our case, we
%   will use Shannon entropy.
% \end{enumerate}
% Here, the universe distribution corresponds to a prior distribution.
% \end{frame}

% \begin{frame}
% \frametitle{Population Diversity with Prior Distribution}
% \bet
% A population that is uniformly sampled from the universe has
%   highest diversity. 
% \eet
% \bet
% Let $S$ be a population sampled within universe $U$, where $P_S$ and
% $P_U$ are the species distributions of $S$ and $U$,
% respectively. Denote $T$ as the set of species. Let $Q$ be a species
% distribution such that
% \[Q(t)=\frac{P_S(t)(P_U(t))^{-1}}{\sum_{v\in T} P_S(v)(P_U(v))^{-1}}.\]
% Then, the diversity of $S$ within $U$ is equal to $H(Q)$.
% \eet
% \end{frame}

% \begin{frame}
% \frametitle{Jensen-Shannon Population Diversity}
% \bed
% We define the Jensen-Shannon Population Diversity of
% $\cP=\{P_1,...,P_k\}$ with respect to prior $P$ as
% $\mbox{JSPD}_P(\cP)=D_{JS}(\Sigma d_i P_i)$ 
% where $d_{i}=\frac{2^{D_{i}}}{\sum 2^{D_{j}}}$ are the normalized
% {\bf exponentiated importances}.
% \eed
% \bet
% Let $\cP$ be such that each $P_i$ has a singleton support. Then, given
% $P$ as the 
% prior,  $JSPD_P(\cP)$ is equal to the diversity of $\cP$ as a
% population within a universe with species distribution $P$.
% \eet
% \end{frame}

\subsection{Text Diversity}

\begin{frame}
\frametitle{Choosing Word Representations}
\begin{itemize}
\item We decided to use Jensen-Shannon Information Diversity, given
  word representations $\cP=\{P_1,...,P_k\}$, and a prior $P$.
\item Given $W=\{w_1,...,w_k\}$, how should the Reader choose $P_{i}$?

{\bf Intuition:} We could use the row of $\cM_{|V|\times|T|}$ that
corresponds to $w_i$, normalized to a distribution.
\item What should the prior $P$ be?

{\bf Intuition:} We can use the overall distribution of topic
assignments from the LDA model.
\end{itemize}
\end{frame}

\begin{frame}
\frametitle{Capturing Topic Similarity}
\begin{itemize}
\item {\bf Problem:} A distributional representation of a word does
  not capture the relations between topics.
\item Suppose we are given a topic similarity matrix
  $\cS_{|T|\times|T|}$, where $\cS_{ij}$ is the number indicating
  the degree of similarity between topics $i$ and $j$. 
\item Let $\tilde{\cS}$ be the matrix $\cS$ with all the rows normalized as
  distributions. 
\item Given $P_i$, we can now consider $\tilde{P_i}=P_i\tilde{S}^T$, which
  essentially diffuses $P_i$ accross similar topics.
\end{itemize}
\end{frame}

\begin{frame}
\frametitle{Capturing Topic Similarity, cont.}
\begin{center}
\begin{tabular}{c|ccc}
%\rowcolor{yellow} \cellcolor{lightyellow}
$\cP$& {\it Phones} & {\it Computers} & {\it Movies}\\
\hline
{\bf Smartphone} & 0.9 & 0.1 & 0\\
 {\bf Mouse} & 0 & 0.9 & 0.1\\
 {\bf Batman} & 0.1 & 0 & 0.9
\end{tabular}
$$\times$$
\begin{tabular}{c|ccc}
$\cS^T$& {\it Phones} & {\it Computers} & {\it Movies}\\
\hline
{\it Phones} & 0.7 & 0.15 & 0.025\\
\hline
{\it Computers} & 0.25 & 0.8 & 0.025\\
\hline
{\it Movies} & 0.05 & 0.05 &  0.95 
\end{tabular}
\end{center}

\end{frame}

\begin{frame}
\frametitle{Capturing Topic Similarity, cont.}
\begin{itemize}
\item We can think of $\tilde{P_i}$ as Reader's {\em second order}
  topic correlation distribution for a given word.
\item To obtain the appropriate prior as average of the word
  representations, we can simply diffuse the old prior
  $\tilde{P}=P\tilde{S}^T$. 
\item To choose a good matrix $\cS$, we can use the topic model. For
  example, we can apply cosine similarity between topic document
  vectors based on the LDA. 
\end{itemize}
\end{frame}

\begin{frame}
\frametitle{Sample Size Bias}
\begin{itemize}
\item {\bf Problem:} Some of the row vectors in $\cM$ may not have
  enough data points to properly estimate the topic distribution.
\item We can treat the Reader as a Bayesian that has a Dirichlet
  prior (concentrated around the prior topic distribution
  $\tilde{P}$). 
\item This leads to a simple smoothing operation:
\begin{equation}
\widehat{P}_i=\frac{\alpha \tilde{P}+ \mu_i \tilde{P}_i}{\alpha+\mu_i}.
\end{equation}
\item Here, $\mu_i$ is the frequency of word $w_i$ in the LDA corpus,
  and $\alpha$ is the prior strength.
\end{itemize}
\end{frame}

\begin{frame}
\frametitle{Conditioning on the Context}
\begin{itemize}
 \item {\bf Problem:} A word $w_i$ can have a significantly different
   meaning in $W=\{w_1,...,w_k\}$, than out of context.
\item {\bf Example:} {\it Mickey Mouse} vs {\it Computer Mouse}.
\item {\bf Solution:} Use words in the context to form a {\em context
    distribution}, and use it to adjust the representation of $w_i$.
% \item Denote $W_{\bar{i}}=W-\{w_i\}$. By $\widehat{P}_{W_{\bar{i}}}$, we denote
%   the mixture distribution $\Sigma d_j\widehat{P}_j$ for $W_{\bar{i}}$
%   (as defined for JSID).
\item Suppose that matrix $\cM$ was generated from the same topics, however
  distributed not as $\tilde{P}$ but as
  $\widehat{P}_{W_{\bar{i}}}$ (the context).
\end{itemize}
\end{frame}

\begin{frame}
\frametitle{Conditioning on the Context, cont.}
\bet
Let $\tilde{P},\widehat{P}_{i}, \widehat{P}_{W_{\bar{i}}}$ be the
topic prior, the general
topic distribution for $w_i$, and the context distribution,
respectively. Then, the context-dependent distribution of $w_i$ will
be 
\vspace{-2mm}
\begin{equation*}
P^{W_{\bar{i}}}_{i}(t)\propto \frac{\widehat{P}_{W_{\bar{i}}}(t)}{\tilde{P}(t)}\widehat{P}_{i}(t).
\end{equation*}
\eet
\bed
We use smoothing again, to obtain the Reader's distribution:
\vspace{-2mm}
\[\widehat{P}^{W_{\bar{i}}}_{i}(t)\propto \beta \widehat{P}_{i}\!(t)
+\frac{\widehat{P}_{W_{\bar{i}}}(t)}{\tilde{P}(t)}\widehat{P}_{i}(t).\] 
\eed
\end{frame}

\section{Experiments}

\begin{frame}
\frametitle{Data Sets}
\begin{enumerate}
\item Interesting phone cases on eBay.
  \begin{itemize}
  \item We hired workers from {\em AMT} to label a collection
of phone cases found on {\em Pinterest} and {\em
  eBay}.
\item The final data-set consists of 2179 positive and 9770 negative instances for
a total of 11,949 instances, with product title used as input. 
\item The topic model was trained using Mallet on 2 million product
  titles, grouped by categories into 800 documents.
\end{itemize}
\item NSF proposal abstracts (61,902 total, 200-300 words each).
\begin{itemize}
\item We used this set 
for training a topic model.
\item To get labeled data, we had to generate $5000$ artificial
  labeled examples, by mixing random abstracts. 
\end{itemize}
\end{enumerate}
\end{frame}

\subsection{Unsupervised Evaluation}

\begin{frame}
\frametitle{Unsupervised Evaluation}
\begin{itemize}
\item We draw an ROC curve by thresholding diversity
  over labeled examples. 
\item As baselines, we use probabilistic diversity
  measures with input being a topic distribution for  given example,
  inferred from the LDA model using Mallet.
\item Baseline measures:
\begin{enumerate}
\item Shannon Entropy - commonly used measure of diversity.
\item Shannon Entropy with topic distributions diffused using the topic
  similarity matrix. 
\item Rao Diversity, (also uses a topic similarity matrix). 
\end{enumerate}
\end{itemize}
\end{frame}

\begin{frame}
\frametitle{Unsupervised Results: eBay}
\begin{figure}
        \centering
        \begin{subfigure}[b]{0.45\textwidth}
                \centering
                \includegraphics[width=40mm]{figures/phonecases-comparison-kopia.png}
               \caption{eBay (baseline)}
                \label{fig:phonecases-comparison}
        \end{subfigure}%\qquad
              ~ %add desired spacing between images, e. g. ~, \quad, \qquad, \hfill etc.
          %(or a blank line to force the subfigure onto a new line)
        \begin{subfigure}[b]{0.45\textwidth}
                \centering
                \includegraphics[width=40mm]{figures/phonecases-breakdown-kopia.png}
                \caption{eBay (JSD)}
                \label{fig:phonecases-breakdown}
        \end{subfigure}\nobreak
\end{figure}
\end{frame}

\begin{frame}
\frametitle{Unsupervised Results: NSF Abstracts}
\begin{figure}
        \begin{subfigure}[b]{0.45\textwidth}
                \centering
                \includegraphics[width=40mm]{figures/nsf-comparison-kopia.png}
                \caption{NSF (baseline)}
                \label{fig:nsf-comparison}
        \end{subfigure}%\qquad
        ~ %add desired spacing between images, e. g. ~, \quad, \qquad, \hfill etc.
          %(or a blank line to force the subfigure onto a new line)
        \begin{subfigure}[b]{0.45\textwidth}
        	        \centering
                \includegraphics[width=40mm]{figures/nsf-breakdown-kopia.png}
               \caption{NSF (JSD)}
                \label{fig:nsf-breakdown}
        \end{subfigure}
\end{figure}
\end{frame}

\subsection{Supervised Evaluation}

\begin{frame}
\frametitle{Supervised Evaluation}
\begin{itemize}
\item We use an unnormalized form of the mixture topic distribution,
  $\tilde{M}=\Sigma D_i\hat{P}_i^{W_{\bar{i}}}$, as a feature vector
  for an SVM classifier trained on eBay data.
\item The results are averaged over five cross-validation $60\%$
  splits.
\item Baselines:
\begin{enumerate}
\item {\em Recursive Auto-Encoders} with Word2Vec - a deep learning approach
  (300 dimensions).
\item {\em Latent Semantic Indexing} - forming a document-term matrix
  and performing SVD. (200 dimensions).
\end{enumerate}
\end{itemize}
\end{frame}

\begin{frame}
\frametitle{Supervised Results}
\begin{table}[t]
%\caption{Classification results for the eBay dataset.}
\label{tab:classification-results}
\vspace{-4mm}
\begin{center}
\begin{tabular}{|l|c|c|c|c|}
\hline
&Precision & Recall & F1 & Accuracy
\\ \hline 
JSID         &$\mathbf{0.714}$&$0.597$&$0.650$& $\mathbf{0.8828}$\\
RAE             &$0.676$&$\mathbf{0.666}$&$\mathbf{0.671}$&$0.8809$ \\
LSI             &$0.676$&$0.633$&$0.654$&$0.8778$\\
\hline
\end{tabular}
\end{center}
\end{table}
\begin{itemize}
\item Our method outperforms the baselines on both accuracy and
  precision.
\item Note, that in this type of problem precision is much more
  important than recall.
\item We obtain a good feature selection technique as a by-product
  of our diversity measure.
\end{itemize}
\end{frame}

\subsection{Conclusions}

\begin{frame}
\frametitle{Conclusions}
\begin{itemize}
\item We proposed a new {\bf diversity measure}, that generalizes entropy.
\item We showed, how it can be applied to {\bf text data}.
\item The results show that the measure performs well for text of
  various lengths, particularly when the {\bf text is very short}.
\item As a by-product of the approach, we get {\bf feature vectors} that
  performed well in a classification task.
\end{itemize}
\end{frame}

\begin{frame}
\frametitle{Future Work}
\begin{itemize}
\item Explore interestingness retrieval in other collections of short pieces of
  text (e.g. Twitter messages).
\item Try JSID feature vectors on other classification tasks.
\item Gain a deeper theoretical understanding of JSID.
\item Find other applications for measuring diversity of a set of
  distributions.
\end{itemize}
\end{frame}

\begin{frame}

\centering{\bf \LARGE Thank you}

\end{frame}

\end{document}