With the rapid growth of e-commerce, new products are increasingly populated into the market place on daily basis.  A larger subset of these products consists of our daily needs or off-the-shelf products, while a much smaller subset can be attributed as {\em unique}, {\em creative}, {\em serendipitous}, or {\em interesting} (see Figure~\ref{fig:ebay-products}). This class of products often provoke an emotive response in users and create a more engaging experience for the them (see  {\em Pinterest} for example). Automatic discovery of this type of products is an important problem in e-commerce for creating an engaging experience for the users.   Quantifying interestingness, however,  is a challenging problem. There has been considerable research on visual interestingness and aesthetic quality of images~\cite{Datta:2006:SAP:2129560.2129588,Ke:2006:DHF:1153170.1153495,IsolaParikhTorralbaOliva2011,dhar:2011,reinecke2013predicting,journals/pami/WeinshallZHKOABGNPHP12}. 
In text domain, researchers have studied different dimensions of this
problem in terms of {\em humor
  identification}~\cite{Mihalcea:2005:MCL:1220575.1220642,Davidov:2010:SRS:1870568.1870582,Kiddon11,labutov-lipson:2012:ACL2012short},
{\em text
  aesthetics}~\cite{journals:tamd:Schmidhuber10,N13-1118,ganguly:2014},
and {\em document diversity}~\cite{bache:2013}.  In this paper we only
focus on text. Our hypothesis is, many interesting texts often
present diversity in the text describing them. In examples shown in Figure \ref{fig:ebay-products}, we have highlighted words that offer the largest diversity in each case.
For example in Figure~\ref{fig:eyeshadow-iphone-case}, in the context of iPhone cases, one would expect less to observe topics that relate to makeup. In this paper we present an information-theoretic approach for measuring topic diversity based on {\em Jensen-Shannon Information Diversity} and show how it correlates with text interestingness. Measuring topic diversity in text has been previously studied by~\cite{bache:2013}. We show how our method differs from this approach and present empirical results over two different data sets: a collection of products from {\sl eBay}, and a corpus of {\sl NSF} proposals. 
