In the previous section we discussed the general setting leading to
obtaining distributional representations of data. However, focusing on
the text domain allows us to extend that model, obtaining more
accurate distributional representations of words. Recall that we start
with a text corpus, which can be converted into a word-context
co-occurrence matrix $\cM_{|V|\times|T|}$. 
We imagine an entity
called Reader, that has observed all of this data, trying to learn the
language from context assignments.
%  First, the Reader estimates the
% distributions of words $P_t(w)$ conditioned on each 
% label $t$, corresponding to the columns of matrix $\Mb$, and also
% the overall distribution of labels $P(t)$ across the entire dataset of 
% co-occurrences. 
% {\bf Do we want to discuss using a prior here? In experiments we don't
%   do that.}
We use a simple generative model $(P,\{P_t\}_{t\in T})$ of the corpus: 
generate a context $t$ from a context distribution $P$, then sample a
word from the word distribution $P_t$,
and repeat to obtain all of the words. We let the Reader estimate
those distributions by maximum likelihood, obtaining the estimated
model $(\tilde{P},\{\tilde{P}_t\}_{t\in T})$.

Next, the question is what should be  
the Reader's distributional representation of a certain word? This
essentially corresponds to a row of matrix $\cM$. A reasonable choice
would be to estimate the MAP context distribution conditioned on a word
$w$ using the appropriate row of $\cM$ and a Dirichlet prior centered
around $\tilde{P}$. However, we propose
a more flexible definition, in which one can obtain a different
distributional representation for the same word, depending on the
context. Suppose that we are given such context in the form of a context
distribution $P_c$. As a thought experiment, consider regenerating our
data using $P_c$ in place of $P$ as the overall context
distribution (obtaining a different co-occurrence matrix $\cM'$), and
estimating the MAP label distribution for $w$ again.
\bep\label{map}
Given a set of word distributions $\{\tilde{P}_t\}_{t\in T}$ for each
context, a context
distribution $P_c$, and a prior context distribution $\widehat{P}$ for a word $w$,
the MAP context distribution for $w$ in the above model will be
\[\Pb(t|w,\widehat{P},P_c) \propto \tilde{P}_t(w)\cdot P_c(t) + \alpha
\widehat{P}(t),\]
where $\alpha$ is the parameter that specifies the strength of the prior.
\eep

% This definition has a nice interpretation through the generative
% model: it corresponds to regenerating our data using $P_c$ in place of
% $P$ as the overall label distribution, and computing an MAP label
% distribution conditioned on $w$, with a Dirichlet prior concentrated around
% $\widehat{P}$.
The most basic application of this is to simply set
both $P_c$ and $\widehat{P}$ equal to $\tilde{P}$ (i.e. there is no extrenal context
information). Then, $P(t|w,\tilde{P},\tilde{P})$ is the
normalized $i$-th row of matrix $\cM$, with added Bayesian
smoothing. These will be the general distributional word
representations, denoted $\tilde{P}_w$. 

The purpose of this model is to provide useful distributional
word representations, that could be applied in various NLP
problems. However, in accordance with the main topic of discussion, we will
focus on the task of measuring diversity of a given sequence of words
$S=(w_1,...,w_k)$. Naturally, we aim to use Jensen-Shannon Diversity
and it seems logical to apply it to some mixture of the set of
distributions $\cP_S=\{\tilde{P}_{w_1},...,\tilde{P}_{w_k}\}$. How do we choose the weights
for the mixture? Following the general model from previous section, we
are tempted to use the frequencies of each of the words in our
corpus. However, that would not  
necessarily match well with the Reader model. Suppose that in our
corpus, word $w_i$ is very common, but it is not well captured by any of
the contexts, and so its context distribution is very close to the overall
context distribution $P$, making it uninformative. For example, in our
experiments the labels come from a topic model, where a stop word may
behave in such a fashion. Of course, that is why stop words are
generally removed from data before topic modeling, but we cannot
expect to filter out every word that might exhibit such behavior. With
that in mind, we will once again turn to KL-divergence to
estimate how informative a word is in a given context.

\bed
We define the {\bf importance} of a distribution $\Pb(t|w,\widehat{P},P_c)$ (in 
the given context) as
\[D(\Pb(\cdot\,|w,\widehat{P},P_c)) =
D_{KL}(\Pb(\cdot\,|w,\widehat{P},P_c)\,\|\,P_c).\]
Given a set of distributions $\cP = \{P_1,...,P_k\}$, obtained from the
Reader model (possibly with different contexts), having importances
$D_i=D(P_i)$, the {\bf Informative Mixture} of $\cP$ is defined as
\[M_{\cP} = \Sigma d_i P_i,\]
where
\[d_i = \frac{D_i}{\sum_{j=1}^k D_j}.\]
\eed

Note, that as in the above definition we will continue to use context
distributions $P_c$ as implicit parameters for brevity of notation.  

% \bep
% Suppose that $\cM$ was generated from the process
% $(P,\{P_t\}_{t\in \cT})$, with $N$ total words. Moreover,
% assume that $N\geq ...$ and let 
% $w$ be a word such that \[P_{t_1}(w)=P_{t_2}(w)=p_w>0\] for all
% $t_1,t_2\in \cT$. If we use prior strength $\alpha = .../\delta^3$, then with
% probability $1-\zeta$ (over the generation of $\cM$), every
% sequence $S=(w_1,...,w_k,w)$ has the property that 
% \[D_{JS}(M_{\cP_{S'}})-...\leq D_{JS}(M_{\cP_S})\leq
% D_{JS}(M_{\cP_{S'}})+...,\]
% where $S' = (w_1,...,w_k)$.
% \eep
% \proof
% Notice that in the given process, the topic distribution conditioned
% on the word $w$ is equal to the overall topic distribution $P$,
% because
% \[\Pb(t|w) \propto P(t)\cdot P_t(w)=P(t)\cdot p_w\propto P(t),\]
% which means that $w$ is completely uninformative with respect to the
% topics, so, ideally it should have no impact on our diversity measure.
% However, we don't necessarily have a good estimate of this
% distribution from the data. Denote $\tilde{P}$ as the data estimate of
% $P$ and $\tilde{P}_w$ as the estimate of $\Pb(t|w)=P(t)$. We can
% bound the error $\tilde{P}(t)-P(t)$ by looking at each column sum of
% matrix $\cM$ as a binomial random variable, obtaining that with
% probability $1-\epsilon_1$, for any $t\in \cT$ we have 
% \[|\tilde{P}(t)-P(t)| < \delta.\]
% We will denote the random error vector as
% $\tilde{\delta}(t)=\tilde{P}(t)-P(t)$. 
% Let $K_w$ be the number of
% occurrences of word $w$ in the data.
% Since $K_w$ could be very small, bounding $\tilde{P}_w-P(t)$ is more
% problematic. Using Proposition \ref{map}, we can write $\tilde{P}_w$ as
% \begin{align*}
% \tilde{P}_w(t) &= \frac{K_w(P(t)+\tilde{\delta}_w^0(t)) +
% \alpha(P(t)+\tilde{\delta}(t))}{K_w+\alpha} \\
% &= P(t) + \frac{K_w\tilde{\delta}_w^0(t) + \alpha \tilde{\delta}(t)}{K_w + \alpha}. 
% \end{align*}
% Setting $\tilde{\delta}_w(t)=\tilde{P}_w(t)-P(t)$, we have two cases:
% \begin{enumerate}
% \item If $K_w>\alpha\delta_w$, then with probability $1-\epsilon_2$ we can bound
% $\tilde{\delta}_w^0(t)$, obtaining
% \[|\tilde{\delta}_w(t)|\leq\max(|\tilde{\delta}_w^0(t)|,|\tilde{\delta}(t)|)
% < \delta_w.\] 
% \item Otherwise, the prior strength $\alpha$ dominates, so we have
% \[|\tilde{\delta}_w(t)|\leq\max(|\tilde{\delta}_w|,|\tilde{\delta}(t)|) < \delta_w.\]
% \end{enumerate}
% Next, we have to bound the importance of word $w$. From the definition
% of KL-Divergence, we obtain the bound
% \begin{align*}
% D_w &= D_{KL}(\tilde{P}_w\,\|\,\tilde{P})=
% D_{KL}(P+\tilde{\delta}_w\,\|\, P+\tilde{\delta}) \\
% &\leq
% (\delta_w+\delta)\frac{T\delta_w+1}{p-\delta}.
% \end{align*}
% Now, we are ready to bound the diversity. Setting 
% $d_w = D_w/(\sum_{v\in S}D_v)$, we have (compensation identity cite) 
% \begin{align*}
% D_{JS}(M_{P_S})&=(1-d_w)(D_{JS}(M_{P_{S'}}) +
% D_{KL}(M_{P_{S'}}\|M_{P_S})) \\
% &+ d_wD_{KL}(\tilde{P}_w\,\|\,M_{P_{S'}}).
% \end{align*}
% Two cases:
% \begin{enumerate}
% \item If $d_w<\delta_2$, then the lower bound is
% \begin{align*}
% D_{JS}(M_{P_S})&\geq (1-d_w)D_{JS}(M_{P_{S'}})\\
% &>D_{JS}(M_{P_{S'}}) - \delta_2\log(T) 
% \end{align*}
% To show the upper bound we use two inequalities:
% \begin{align*}
% D_{KL}(M_{P_{S'}}\|M_{P_S})&\leq ...\\
% d_w D_{KL}(\tilde{P}_w\|M_{P_{S'}})&\leq ...=d_w\log\frac{1}{d_w}
% \end{align*}
% Combining the above inequalities, we get:
% \begin{align*}
% D_{JS}(M_{P_S})\leq D_{JS}(M_{P_{S'}}) + ...
% \end{align*}
% \item If $d_w\geq \delta_2$, then (compensation identity again) 
% \begin{align*}
% D_{JS}(M_{P_S})\leq \sum_{v\in
% S}D_{KL}(\tilde{P}_v\|\tilde{P})\leq \frac{D_w}{\delta_2}.
% \end{align*}
% The same holds for $S'$. So, we get both bounds.
% \end{enumerate}
% \qed

We now turn to the contexts themselves, and why a fixed distributional
word representation may not be sufficient for this task. A simple example will
help us illustrate this. Consider a word $w_i$ that has
two separate meanings. In one case, it is always assigned context $t_1$,
while in the other - context $t_2$. The general distributional
representation of $w_i$ will put significant weight on both $t_1$ and
$t_2$, even though in each specific occurrence, $w_i$ represents only one
or the other. A context distribution $P_c$ could help disambiguate
the meaning of $w$. In our case, the context of $w_i$ will be composed of
all other words in the sequence $S$, because in our experiments we
focus on short to medium length pieces of text. Otherwise, a moving
window of fixed length may be a more appropriate choice.

\bed
Given a sequence $S=(w_1,...,w_k)$, we define the context-dependent
distributional representation of word $w_i$ in $S$ by 
\[\tilde{P}_{w_i}^S(t) = P(t|w_i,\tilde{P}_{w_i},\tilde{P}_S^{w_i}),\]
where $\tilde{P}_S^{w_i}$ is the Informative Mixture of
$\{\tilde{P}_{w_j}\}_{j\neq i}$. 
\eed

This gives us a new set $\cP_S^c=\{\tilde{P}_{w_1}^S,...,\tilde{P}_{w_k}^S\}$ of
distributional representations for each word in $S$. Moreover, we can
use them to obtain a distributional representation of $S$ itself, by
taking the Informative Mixture $M_{\cP_S^c}$. As we will show in
experiments, this can be treated separately as a text embedding in
supervised learning tasks. In this section, however, we will use it to
propose a notion of text diversity, based on Definition
\ref{jsd-definition}. 

\bed
Given a sequence $S=(w_1,...,w_k)$ and a data model $\cM$, the
{\bf Jensen-Shannon Text Diversity} will be defined as
\[\textrm{JSD}_{\cM}(S) = D_{JS}(M_{\cP_S^c}),\] where $\cP^c_S=\{P_{w_1}^S,...,P_{w_k}^S\}$ is the set
of context-dependent distributional representations for words in $S$. 
\eed


%  We can measure how much information is gained
% in this observation using KL-divergence, obtaining
% $D_{KL}(P_i\|P)$. Moreover, we can bring in some information about
% relations between words, by computing $P_{ij}(\,\cdot\,) =
% P(\,\cdot\,|w_i,P_i,P_j)$, 
% which describes the topic distribution obtained after observing word
% $w_i$ in the context of word $w_j$ (since they were both in
% $S$). Finally, we define a new series of topic
% distributions for each word, which are based on contexts from all other
% words in the sequence, weighed by their information gain.



% \bed
% Given a co-occurrence matrix $\Mb$, we define the Jensen-Shannon
% Information Diversity of a sequence 
% $S=(w_1,...,w_k)$ as
% \[\mbox{JSID}_{\Mb}(S)=D_{JS}(\Sigma d_i \widetilde{P}_i),\]
% where 
% \[d_{i}=\frac{D_{KL}(\widetilde{P}_i\|P_i^c)}{\sum D_{KL}(\widetilde{P}_i\|P_i^c)}.\]
% \eed

% Notice, that we use different weights than were proposed for the
% population diversity (we use surprise instead of perplexity). To
% best observe the difference, let us see what happens when a given word
% provides no information gain (the divergence from context distribution
% is zero). In the above definition, we would put no weight on that
% word, since it is irrelevant to the meaning of sequence $S$. On the
% other hand, using perplexity means that all weights have to be
% non-zero, which makes more sense in the population
% setting. Unfortunately, the
% use of different weights for the information diversity prevents us
% from obtaining the generalization from Proposition
% \ref{generalization}. However, it is easy to see that JSID is still a
% generalization of Shannon entropy in a more narrow sense, because it
% can be interpreted as mutual information. 

% {\bf (4) Sample size bias problem:}  note that 
% when we normalize the rows of the matrix $\cM$ to obtain a word-to-topic distributional model, the normalization factor 
% for every word is simply the word count. Thus for a word $w$ which occurs very rarely
% in the entire corpus $\cC$, the topic distribution will be artificially skewed and is inaccurate simply because we do not have enough data points to estimate its true word-to-topic distribution.
% Now, if for this corpus it happens that the prior topic distribution
% is close to uniform, it will give a very high importance to the word
% $w$ when measuring the information diversity as described in
% Section~\ref{sec:information-diversity}.
% One way to alleviate this
% problem is to use the relative sample size (e.g., word count) to
% smooth the distribution obtained by normalizing the rows of matrix
% $\cM$. A natural choice for the smoothing distribution would in this
% case be simply the prior distribution mentioned above. Applying {\em Laplace smoothing}
% we get:
% \begin{equation}
% \widehat{P}_w=\frac{\alpha \tilde{P}+ \mu_w \tilde{P}_w}{\alpha+\mu_w}
% \end{equation}
% where $\mu_w$ is the frequency of word $w$ in D, and $P_w$ is the
% topic assignment distribution obtained from the word-topic matrix,
% while $\alpha$ is the parameter that specifies the strength of the
% prior.\\
% {\bf (5) Conditioning on context words:} we propose a final enhancement to the word-topic
% distributions. Suppose, the set of words $W=\{w_1,...,w_k\}$
% represents a text snippet that we want to analyze. The word $w_i$ has
% a specific meaning inside of $W$, that can be significantly different
% than its meaning out of context. Denote
% $W_{\bar{i}}=W-\{w_i\}$ as the set of all words in $W$ except
% $w_i$. By $P_{W_{\bar{i}}}$, we denote the mixture distribution for $W_{\bar{i}}$ (Definition \ref{mixture}) and we use $\tilde{P}_{W_{\bar{i}}}$ when it is smoothed using Laplace smoothing method. We
% propose the following definition of context-dependent word-topic
% distribution:
% \bed
% Let $\tilde{P},\widehat{P}_{w_i}, \tilde{P}_{W_{\bar{i}}}$ be the topic prior, general
% topic distribution for $w_i$, and the context distribution,
% respectively. Then, the context-dependent distribution is
% \begin{equation*}
% P^{W_{\bar{i}}}_{w_i}(t)\propto \frac{\tilde{P}_{W_{\bar{i}}}(t)}{\tilde{P}(t)}\widehat{P}_{w_i}(t)
% \end{equation*}
% \eed
% There is a probabilistic explanation that we have left out for 
% lack of space. However this can be intuitively understood as follows:
% we can think of $\frac{\tilde{P}_{W_{\bar{i}}}(t)}{\tilde{P}(t)}$ as a weight
% that further reshapes the smoothed word-to-topic distribution $\widehat{P}_{w_i}(t)$
% to take into account the context. In our experiments we also smooth this distribution
% using {\em Laplacian} smoothing.
