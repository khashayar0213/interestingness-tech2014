There has been considerable research on visual interestingness and aesthetic quality of images~\cite{Datta:2006:SAP:2129560.2129588,Datta:2008:4233023,Ke:2006:DHF:1153170.1153495,IsolaParikhTorralbaOliva2011,dhar:2011,reinecke2013predicting,journals/pami/WeinshallZHKOABGNPHP12,journals/pami/WeinshallZHKOABGNPHP12}. In~\cite{Datta:2006:SAP:2129560.2129588,Datta:2008:4233023,Ke:2006:DHF:1153170.1153495,dhar:2011} some nice ideas on how people in general judge aesthetic quality of photographs for designing low level features that might correlate with human measures have been presented. \citep{IsolaParikhTorralbaOliva2011} investigate how some images are intrinsically more memorable than others and study a a set of features that describe interpretable spatial, content, and aesthetic image properties. \citep{reinecke2013predicting,journals/pami/WeinshallZHKOABGNPHP12}  study the problem of predicting the initial impression of aesthetics
based on perceptual models of a website's colorfulness and visual complexity in terms of a set of low level features describing such properties on a web site.  
\citep{journals/pami/WeinshallZHKOABGNPHP12} define an approach based on distinct types of unexpected events when general-level and specific-level classifiers give conflicting predictions for novelty detection in videos and images. In spirit, our approach has the same flavor as \cite{journals/pami/WeinshallZHKOABGNPHP12}  in that we are looking for unexpected combination of topics to identify interestingness in text.

 
There has been also a large body of work in text domain. Researchers have studied different dimensions of this problem in terms of {\em humor identification}~\cite{Mihalcea:2005:MCL:1220575.1220642,Davidov:2010:SRS:1870568.1870582,Kiddon11,labutov-lipson:2012:ACL2012short},
{\em text aesthetics}~\cite{journals:tamd:Schmidhuber10,N13-1118,ganguly:2014}, and {\em document diversity}~\cite{bache:2013}.  \citep{Mihalcea:2005:MCL:1220575.1220642} study a computational approach for humor recognition by utilizing a set of humor-specific stylistic features such as alliteration, antonymy, and adult slangs. Some of these features, e.g., antonymy, in a limited way capture some sort of text diversity as we do not normally expect antonyms co-occurring in standard text. \citep{Davidov:2010:SRS:1870568.1870582}  propose a semi-supervised approach for identifying sarcastic sentences
in Twitter and Amazon.  Their approach consists of two stages; a semi supervised pattern acquisition is used for identifying
sarcastic patterns, and classifier that uses such patterns as features in a classification task. \citep{Kiddon11} propose an approach for double entendre identification by utilizing  a set of word-level and phrase-level analytical features (noun sexiness, adjective sexiness, and verb sexiness) in a classification task. Such word-level stand-alone interestingness features are related to the word-saliency factor that is discussed in Section~\ref{sec:the-readers-model}. \citep{ganguly:2014} study automatic prediction of text interestingness by utilizing a broader set of features such as word length, repetitions, polarity, part-of-speech, semantic distances, and somewhat simple treatment of topic generality and diversity.

Measuring topic diversity for text has been previously studied by~\cite{bache:2013} which is the most relevant work to our approach. \citep{bache:2013} uses Rao's diversity \cite{rao:1982} for measuring document diversity based on a topic model learned over a corpus of documents. As it will be cleared in Sections~\ref{sec:information-diversity} through~\ref{sec:topic-similarities} our measure differs from this metric from an information theoretic perspective. Throughout experiments presented in Section~\ref{sec:experiments} we use this approach as one of our baselines.